% LaTeX document
\documentclass[11pt]{article}
\usepackage[spanish,activeacute]{babel}
\usepackage{graphicx}
\usepackage{epsfig}
\usepackage{subfigure}
\usepackage{url}
\usepackage[colorlinks=true]{hyperref}

\setlength{\topmargin}{-.5in}
\setlength{\textheight}{9in}
\setlength{\oddsidemargin}{.125in}
\setlength{\textwidth}{6.25in}
%\renewcommand{\familydefault}{\sfdefault}


\begin{document}

\title{Sistemas de recomendaci'on: conceptos y prototipo}
%\title{Evaluaci'on de niveles de satisfacci'on en sistemas de recomendaci'on: estado del arte}
\author{blancavg}


\maketitle
\tableofcontents
\newpage
%%%%%%%%%%%%%%%%%%%%%%%%%%%%%%%%%%%%%%%%%%%%%%%%%%%%%%%%
\abstract{
En este documento se resumen los conceptos generales revisados en diversos \textit{surveys} del estado del arte. El objetivo es conocer la terminolog'ia y conocer los distintos enfoques usados en la construcci'on de sistemas de recomendaci'on. Esta revisi'on servir'a de base a un estudio detallado sobre metodolog'ias de evaluaci'on de niveles de satisfacci'on en sistemas de recomendaci'on.
}
%%%%%%%%%%%%%%%%%%%%%%%%%%%%%%%%%%%%%%%%%%%%%%%%%%%%%%%%
\section{Introducci'on}
%Esta secci'on se basa en los surveys:~\cite{recsys:alban,start:candillier09,recsys:nlathia,almazrorecsys}.

Los primeros art'iculos sobre sistemas de recomendaci'on aparecieron a mediados de los noventa. Estos sistemas tienen por objetivo ayudar a los usuarios a encontrar 'items que sean de su agrado de entre numerosos art'iculos en un cat'alogo. Los 'items pueden ser de cualquier tipo: pel'iculas, m'usica, libros, sitios web, noticias, restaurantes, estilos de vida. Los sistemas de recomendaci'on ayudan a los usuarios a encontrar 'items de su inter'es bas'andose en alguna informaci'on sobre sus preferencias hist'oricas ~\cite{start:candillier09}.

La gran cantidad de recursos existentes hace muy dif'icil que el usuario los vea todos, eso sin contar con que se genera informaci'on nueva constantemente. Los sistemas de recomendaci'on son herramientas para la ayuda de toma de decisiones sin necesitar una b'usqueda.

Tareas comunes para los sistemas de recomendaci'on pueden ser: encontrar buenos 'items, encontrar todos los 'items, recomendar una secuencia de 'items o como ayuda de navegaci'on. Sin embargo, el objetivo principal es filtrar contenido para proporcionar sugerencias relevantes y 'utiles para cada usuario del sistema. Los sistemas de recomendaci'on difieren de la tradicional recuperaci'on de informaci'on construyendo modelos para las preferencias del usuario y combinando selectivamente las opiniones de usuarios distintos para generar recomendaciones 'unicas por usuario~\cite{recsys:nlathia}. Existen dos tipos de problemas con los datos:
\begin{itemize}
\item Recuperaci'on de informaci'on (IR): IR es la ciencia de la b'usqueda de documentos. contenido est'atico, query din'amico $\rightarrow$ modelado de contenido (organizado con 'indices).
\item Filtrado de informaci'on (IF): Los sistemas de IF eliminan informaci'on redundante o innecesaria de un flujo de informaci'on. Contenido din'amico, query est'atico $\rightarrow$ modelado de query (organizado como filtros).

La recomendaci'on se encuentra entre IR e IF ya que el contenido var'ia suavemente y los queries dependen de pocos par'ametros. Los m'etodos para IR e IF son usados para reducir el proceso en tiempo de query~\cite{recsys:alban}.
\end{itemize}

\subsection{Terminolog'ia b'asica}

\smallskip
\textbf{Usuario:} es el usuario final del sistema, a quien se le quiere recomendar algo.

\smallskip
\textbf{Rating:} es la medida que describe qu'e tanto le gustar'a al usuario un 'item espec'ifico. Generar recomendaciones a menudo se describe como el problema de predecir el rating del usuario sobre un 'item. Los ratings pueden ser expl'icitos o impl'icitos.

\smallskip
\textbf{Perfil:} los usuarios pueden ser modelados de acuerdo a una gran variedad de informaci'on. La informaci'on m'as importante es el conjunto de evaluaciones (ratings) que han dado al sistema, los cuales corresponden a cada perfil de usuario.

\smallskip
\textbf{Relaci'on entre ratings y perfiles:} el punto central de un sistema de recomendaci'on es el conjunto de perfiles de usuario. Los perfiles contienen una colecci'on de juicios o calificaciones del contenido disponible y proporciona una fuente invaluable de informaci'on que puede usarse para dar recomendaciones a los usuarios. Los juicios humanos sin embargo, pueden provenir de dos fuentes separadas:
\begin{enumerate}
\item \textbf{Ratings expl'icitos}. Calificaciones num'ericas dadas por el usuario.
\item \textbf{Ratings impl'icitos}. Son dados por el comportamiento del usuario como por ejemplo, el tiempo leyendo una p'agina, el n'umero de veces que escucha una canci'on o artista o el art'iculo que ha visto al navegar por un cat'alogo.
\end{enumerate}

Los ratings impl'icitos pueden convertirse a un valor num'erico mediante una funci'on apropiada (\textit{transpose function}). El conjunto de juicios por usuario, comparado al n'umero total de 'items que pueden calificarse ser'a muy peque\~{n}o.

\smallskip
\textbf{Matriz de calificaci'on de usuario:} as'i se le conoce al conjunto de perfiles de usuario.

%-------------------------------------------------------
%\medskip
%\textbf{ParaPost} \\
%Pedir y dar recomendaciones son cosas que seguramente se han realizado desde que el hombre apareci'o en la Tierra. Ya me imagino:

%\begin{itemize}
%\item Mejores 'areas de caza.
%\item D'onde hay cuevas, las mejores, las que protegen m'as.
%\item D'onde hay r'ios, lagos o lagunas.
%\item Grupos de personas pac'ificas o conflictivas.
%\end{itemize}

%No ser'ia raro que las pinturas rupestres sean formas primitivas de sistemas de recomendaci'on.

%La recomendaci'on de persona a persona es muy com'un y sigue y seguir'a vigente. Normalmente pedimos recomendaciones a personas con las que coincidimos en algo y sabemos que nos puede hacer recomendaciones que nos agraden.

%Pero, cuando la b'usqueda es en contenidos de internet y se cuenta con mucha informaci'on es cuando los sistemas de recomendaci'on pueden ser de gran ayuda. Desde algo tan simple como \textquestiondown qu'e pel'icula ver'e hoy? hasta informaci'on sobre salud. Con el continuo y r'apido aumento de la cantidad de informaci'on, los sistemas de recomendaci'on han surgido como una herramienta para tomar decisiones. Los recsys filtran contenido para proporcionar sugerencias 'utiles al usuario.


\subsection{Tipos de sistemas de recomendaci'on}

Bas'andose en c'omo se generan las recomendaciones, existen diversos tipos de sistemas de recomendaci'on. Los enfoques m'as simples son los siguientes:
\begin{itemize}
\item \textbf{Algoritmo de predicci'on aleatorio}. Selecciona aleatoriamente 'items de un conjunto de 'items disponibles y los recomienda al usuario. La precisi'on se basa en la suerte y mientras m'as grande es el conjunto, las recomendaciones son peores. Se utiliza s'olo como referencia para comparar la calidad de resultados de algoritmos m'as sofisticados~\cite{almazrorecsys}.
\item \textbf{Secuencias frecuentes}. Se basa en encontrar patrones frecuentes. Por ejemplo, si un cliente califica 'items con regularidad, se puede usar ese patr'on para recomendarle otros. La desventaja es que s'olo es eficiente despu'es de que el cliente ha realizado compras~\cite{almazrorecsys}.
\end{itemize}

Los siguientes tipos de sistema de recomendaci'on son los com'unmente implementados por lo que se explicar'an detalladamente en las siguientes secciones.

\begin{itemize}
\item Filtrado basado en contenido.
\item Filtrado colaborativo (CF).
\item Filtrado h'ibrido.
\end{itemize}

El que predomina es el filtrado colaborativo.

%%%%%%%%%%%%%%%%%%%%%%%%%%%%%%%%%%%%%%%%%%%%%%%%%%%%%%%%
\section{Filtrado basado en contenido}
Estos sistemas hacen una correspondencia entre un 'item y el usuario bas'andose en la descripci'on del 'item y el perfil del usuario.

El punto importante es tener una forma de describir el 'item, descomponer el contenido en atributos enumerables. Estos atributos pueden ser la frecuencia de la palabra o etiquetas dadas por el usuario. Se necesita una forma de construir el perfil para que incluya los tipos de 'item que le gustan y una forma de comparaci'on entre 'item y el perfil. Los 'items que tengan un alto nivel de cercan'ia a las preferencias del usuario se recomendar'an.

El \textbf{perfil de usuario} puede construirse de forma \textit{impl'icita} a partir de las preferencias del usuario para 'items, ya sea buscando los 'items que le han gustado o disgustado o en sus acciones pasadas (historial de compras). Tambi'en de forma \textit{expl'icita} mediante cuestionarios acerca de las descripciones del 'item.

Un \textbf{modelo del usuario} puede aprenderse de forma \textbf{impl'icita} usando un m'etodo de aprendizaje autom'atico, tomando como entrada las descripciones del 'item y generando como salida las apreciaciones del usuario sobre el 'item.

Los perfiles del usuario generalmente se representan como vectores de pesos sobre las descripciones el 'item. Las recomendaciones son generadas aplicando m'etodos al modelo de preferencias del usuario. Entre estos m'etodos est'a la inducci'on de reglas y 'arboles de decisi'on.

Las preferencias indican la relaci'on entre un usuario y los datos. La cobertura de una preferencia se relaciona directamente a la cobertura del o los atributos a los cuales se aplica. Un atributo tiene una alta cobertura cuando aparece en muchos 'items y una baja cobertura si aparece en pocos. Sin embargo, la cobertura puede extenderse usando la noci'on de similaridad entre atributos. Si la preferencia de un usuario es \textit{Me gusta Jackie Chan como actor}, la cobertura es baja pero si se incluye \textit{Me gusta Jackie Chan como actor} como son considerados de forma similar, la cobertura se extiende, esto suponiendo que no existen contradicciones con otras preferencias.

 Diversos enfoques se han seguido para determinar la similaridad entre atributos. Tradicionalmente 'esto es realizado por un experto. 'Esto es popular para dominios peque\~{n}os pero para grandes, es impr'actico. Como alternativa, hay medidas de similaridad para tomar ventaja de la riqueza de informaci'on existente en internet. Una de ellas es la NormalisedGoogleDistance, que infiere similaridades entre t'erminos textuales usando co-ocurrencia en websites. Sin embargo, para grandes bases de datos esto tambi'en es d'ebil.

Para evitar estas restricciones, se han preferido las m'etricas de similaridad que analizan directamente las bases de datos de los sistemas de recomendaci'on. Por ejemplo, se crean vectores para dos actores y se les aplica la medida wCosine para compararlos.

Otros enfoces son usando clasificadores como Naive Bayes, dando como entrada las descripciones del 'item y como salida los gustos del usuario para un subconjunto de 'items. El clasificador se entrena sobre un conjunto de 'items ya considerados por el usuario. As'i es capaz de predecir si un nuevo elemento le gustar'a o no al usuario.

\medskip
\noindent
\textbf{Ventajas}

\begin{itemize}
\item Usando aprendizaje inductivo con 'arboles y reglas se pueden adaptar r'apidamente y cambiar recomendaciones bas'andose en la retroalimentaci'on del usuario. 'Esto lleva a la idea de recomendadores conversacionales que permiten a los usuarios revisar las preferencias que dan como entrada criticando los resultados obtenidos. Los modelos del usuario son din'amicos y permiten a los usuarios entender el efecto de sus preferencias en las recomendaciones recibidas.
\item Cubre limitaciones de los colaborativos: pueden generar recomendaciones para nuevos 'items sin necesidad de ratings disponibles.
\item Pueden manejar situaciones donde los usuarios no consideran los mismos 'items pero si 'items parecidos.
\end{itemize}

\medskip
\noindent
\textbf{Desventajas}

\begin{itemize}
\item Necesitan descripciones ricas y completas de 'items y perfiles de usuario bien construidos. 'Esta es la principal limitaci'on.
\item Sufren de sobre-especializaci'on, i.e., a menudo recomiendan 'items de contenido similar a los 'items ya considerados lo que puede llevar a una falta de originalidad.
\item Las decisiones de los usuarios van mas all'a de lo que puede representarse en t'erminos de atributos.
\item No se pueden aplicar a todo el rango de escenarios.
\item Requieren que el contenido pueda ser descrito en t'erminos de atributos cosa que no siempre es posible.
\item A menudo se requieren grandes cantidades de detalles del usuario para hacer buenas recomendaciones.
\end{itemize}


%%%%%%%%%%%%%%%%%%%%%%%%%%%%%%%%%%%%%%%%%%%%%%%%%%%%%%%%
\section{Filtrado colaborativo}
Seg'un~\cite{recsys:nlathia}:\\
Los algoritmos de filtrado colaborativo, a diferencia de los sistemas basados en contenido, ignoran totalmente cualquier descripci'on o atributos de los datos. Favorecen los juicios humanos y se enfocan en generar las recomendaciones con base en las opiniones expresadas por una comunidad de usuarios. Han sido usados ampliamente en una gran diversidad de sitios web.

La generaci'on de recomendaciones y el uso de datos disponibles se ha abordado desde distintas perspectivas. Cada una aplica diferentes heur'isticas y metodolog'ias para crear recomendaciones. Se revisan dos grandes categor'ias de filtros: basado en la memoria y basado en el modelo, posteriormente se da un vistazo a otros m'etodos y enfoques h'ibridos.


%%%%%%%%%%%%%%%%%%%%%%%%%%%%%%%%%%%%%%%%%%%%%%%%%%%%%%%%
\subsection{Basado en memoria}
A menudo es referido como el m'etodo dominante para generaci'on de recomendaciones. Su clara estructura junto con sus buenos resultados lo hace una f'acil selecci'on. Se le llama basado en memoria porque supone que los usuarios que han pensado de forma parecida anteriormente, continuar'an compartiendo sus intereses en el futuro. Por lo tanto, las recomendaciones para un usuario pueden generarse prediciendo ratings de contenido no calificado, bas'andose en una agregaci'on de ratinga dados por usuarios parecidos (o cercanos) de la misma comunidad. Por esta raz'on, al proceso se le conoce como kNN, o filtro de los k vecinos m'as cercanos y consiste en tres fases: formaci'on del vecindario, agregaci'on de la opini'on y recomendaci'on.

\begin{enumerate}
\item \textbf{Formaci'on del vecindario:} La idea es encontrar un subconjunto 'unico de la comunidad para cada usuario, identificando a otros usuarios con intereses similares que puedan actuar como recomendadores. Para hacerlo, cada pareja de perfiles de usuario se compara para medir el grado de similaridad $w_{a,b}$ compartido. En general, el rango de similaridad var'ia desde $1$ (perfecta similaridad) a $-1$ perfecta disimilaridad. Si un par de usuarios no tiene coincidencias, no hay forma de comparar su similaridad la cual se pone a $0$.

La similaridad puede medirse de diferentes maneras. El objetivo es modelar la relaci'on potencial entre usuarios con un valor num'erico. Veamos varias:

\begin{itemize}
\item La medida m'as simple para medir la fuerza de la relaci'on es contar la proporci'on de 'items co-calificados compartidos por un par de usuarios: poner ecuaci'on. Esta medida ignora los valores de las calificaciones y s'olo considera lo que cada usuario ha calificado; es el tama\~{n}o de la intersecci'on de los dos perfiles del usuario sobre el tama\~{n}o de la uni'on. La suposici'on es que dos usuarios que cont'inuamente califican los mismos 'items comparten esa caracter'istica com'un.
\item El m'etodo m'as citado para medir la similaridad es el Coeficiente de Correlaci'on de Pearson que mide el grado de linearidad que existe en la intersecci'on de un par de usuarios. poner ecuaci'on. Fu'e la primera que se propuso y es el coseno de las desviaciones de la media. Otras que son tradicionales son el Coseno simple y Manhattan.

La desventaja de estas medidas es que s'olo se consideran los atributos en com'un entre dos vectores. As'i, los vectores ser'an id'enticos aunque s'olo compartan una evaluaci'on en un atributo.
\item La similaridad de \textit{Jaccard} no sufre de esta limitaci'on puesto que mide la sobreposici'on que dos vectores comparten. Lo malo es que no toma en cuenta la diferencia de ratings por lo que si dos usuarios eval'uan por ejemplo, las mismas pel'iculas pero con calificaciones opuestas, se consideran similares.
\item Al combinar Jaccard con otras medidas, se obtienen mejores beneficios. Por ejemplo, \textit{wPearson} es una medida de Pearson pesada. De forma parecida, \textit{wCosine} y \textit{wManhattan} son una combinaci'on de Jaccard con Cosine y Manhattan respectivamente.
\item Una mejora al Coeficiente de Pearson es el pesado de significancia: si el n'umero de 'items co-calificados $n$ es menor que un umbral $x$, la medida de similaridad es multiplicada por $n/x$. La medida es m'as confiable conforme el n'umero de 'items co-calificados aumenta.
\item Otra modificaci'on es el Coeficiente de Pearson con restricciones, que reemplaza la media del usuario en la ecuaci'on con el \textit{rating scale midpoint}.
\item En el pasado se han usado otras medidas de similaridad, una de ellas es la \textit{Spearman Rank Correlation}.
\item Vector Similarity (o $cosine angle$ entre dos perfiles de usuario).
\item Distancia Euclidiana y Manhattan.
\item Otros m'etodos que tratan de capturar la proporci'on de acuerdo entre usuarios tales como los de Atresty y Winner (1997).
\end{itemize}

Las medidas de similaridad permanecen como un 'area abierta ya que no se puede hacer m'as que comparar la exactitud de la predicci'on para demostrar que una medida es mejor que otra en un conjunto particular.


\item \textbf{Agregaci'on de opini'on:} una vez que las comparaciones entre el usuario y el resto de la comunidad se realizaron, tenemos un conjunto de pesos de los recomendadores. Con los pesos obtenidos es predecir ratings de contenido no calificado. Al igual que la fase anterior, hay diversas formas de calcular estas predicciones. En dos m'etodos (Herlocker et al, 1999 y Koren, 2007) se predice un rating $p_{a,j}$ del 'item $i$ para el usuario $a$ como el promedio pesado de ratings de vecinos $r_{b,i}$. Los pesos $w_{a,b}$ son la medida de similaridad del paso anterior por lo que los vecinos m'as similares tendr'an mayor influencia en la predicci'on.

\smallskip
\textquestiondown Qu'e ratings son los escogidos para contribuir a predecir el rating? Nuevamente hay mucho de donde escoger que tendr'a un impacto directo en los resultados. En algunos casos, se toman s'olo los $k$-mejores vecinos cercanos para contribuir. Sin embargo, a menudo ninguno de esos vecinos ha calificado al 'item en cuesti'on y as'i, la cobertura de la predicci'on se impacta negativamente. Una alternativa directa, por tanto, es considerar los $k$-mejores recomendadores que disponen de rating para el 'item en cuesti'on. Por un lado, este m'etodo garantiza que se har'a una predicci'on; por el otro las predicciones se hacen con base en usuarios modestamente similares por lo que no pueden ser muy precisas.

\smallskip
Una 'ultima alternativa es seleccionar s'olo los usuarios sobre un umbral de similaridad pre-determinado. Pero, \textquestiondown cu'al ser'ia el umbral o valor de $k$?. Son preguntas sin respuesta y dependen del conjunto de datos.

\item \textbf{Recomendaci'on:} una vez que se han predicho los ratings para los 'items y ordenados de acuerdo al valor de la predicci'on, los $n$-mejores 'items se proponen al usuario final como recomendaciones. Ahora, se puede obtener retroalimentaci'on del usuario. Los perfiles del usuario crecer'an y el sistema puede empezar repetir el proceso: re-calcular medidas de similaridad, predecir ratings y dar recomendaciones. Hasta el momento se ha considerado el proceso de generaci'on de recomendaciones 'unicamente por un enfoque basado-en-memoria de vecinos-cercanos. En otra secci'on se ver'an las contribuciones desde el 'area de aprendizaje autom'atico, a menudo llamadas filtrado colaborativo basado-en-el-modelo.
\end{enumerate}

Para~\cite{start:candillier09}, la entrada al sistema es un conjunto de ratings sobre los 'items. Los usuarios pueden compararse con base en la apreciaci'on que comparten sobre los 'items, creando la noci'on de vecindarios de usuarios. De forma parecida, los 'items pueden compararse con base en la apreciaci'on compartida por los usuarios, formando la noci'on de vecindarios de 'items. Los ratings del 'item para un para un usuario dado pueden predecirse con base en los ratings dados en su vecindario de usuarios y el vecindario de 'items.

\smallskip
\textbf{Formalizaci'on}. Sea $U$ un conjunto de $N$ usuarios, $I$ un conjunto de $M$ 'items, y $R$ un conjunto de ratings $r_{ui}$ de usuarios $u \in U$ en el 'item $i \in I$. $S_u \subseteq I$ significa el conjunto de 'items que el usuario $u$ ha calificado.

El objetivo de los enfoques de filtrado colaborativo es predecir el rating $p_{ai}$ de un usuario $a$ sobre un 'item $i$. Se supone que el usuario $a$ es activo, i.e., ya ha calificado algunos 'items. El 'item a ser predecido es desconocido al usuario $i \notin S_{a}$.

\begin{itemize}
\item \textbf{Enfoque basado en el usuario.} Es igual que el que describe~\cite{recsys:nlathia} en la fase de \textbf{agregaci'on de opini'on} y consiste en predecir el rating de un usuario para un 'item con base en los vecinos cercanos.

\item \textbf{Enfoque basado en el 'item.} Recientemente ha crecido el inter'es por los enfoques basados en el 'item. Dada una medida de similaridad entre 'items, tales enfoques primero definen vecindades de 'items. La predicci'on del rating de un usuario por un 'item se deriva de los ratings del usuario en los vecinos del 'item fijado.

As'i como en los enfoques basados en el usuario, la vecindad de 'items tama\~{n}o $K$ es un par'ametro del sistema que necesita definirse. Dado $T_i$, la vecindad del 'item $i$, se consideran dos formas para predecir nuevos ratings del usuario:

\begin{enumerate}
\item usando una suma pesada
\item usando una suma pesada de las desviaciones de la media de los ratings del 'item
\end{enumerate}
\end{itemize}


%-------------------------------------------------------
\subsection{Basado en el modelo}
De acuerdo a~\cite{recsys:nlathia}, los enfoques basados en modelo que provienen del aprendizaje autom'atico, aplican sus t'ecnicas al problema del fitrado de informaci'on.

El objetivo es predecir cu'anto les gustar'a o calificar'an los usuarios el contenido que aun no eval'uan y rankear esos 'items para proporcionar las $n$-mejores recomendaciones. En otras palabras, el filtrado colaborativo cae entre las categor'ias de \textit{clasificaci'on}, o decidir a qu'e grupo de clasificaci'on pertenecen los 'items no evaluados, y \textit{regresi'on}, el proceso de modelar la relaci'on que una variable (tal como una calificaci'on de usuario) tiene con otras variables (el conjunto de perfiles de usuario). Algunos de los algoritmos aplicados son los siguientes:

\begin{itemize}
\item Algoritmo p-rank (Crammer y Singer, 2001). Aborda el problema como clasificaci'on lineal y se basa en clasificadores perceptr'on.
\item Singular value decomposition.
\item Redes neuronales.
\item Redes Bayesianas.
\item M'aquinas de soporte vectorial.
\item Aprendizaje inductivo de reglas.
\item An'alisis sem'antico.
\end{itemize}

Todos se basan en inferir reglas y patrones a partir de las calificaciones de los datos. Los enfoques basados en el modelo son atractivos porque una vez entrenados las predicciones se generan de forma r'apida y eficiente. Sin embargo, su 'exito es limitado porque los enfoques basados en memoria son m'as simples y son igualmente precisos.

Otra diferencia es la interpretaci'on. Los m'etodos basados en memoria modelan a los usuarios bas'andose en valores de similaridad medibles lo que genera la noci'on de una comunidad de recomendadores. Por el contrario, los enfoques basados en el modelo entrenan un modelo para cada usuario y se caracterizan por una visi'on m'as subjetiva de los usuarios finales del sistema.

La idea general es derivar un modelo off-line de los datos para predecir ratings on-line lo m'as r'apido posible. En~\cite{start:candillier09} se mencionan los siguientes m'etodos usados:

\begin{itemize}
\item El primer tipo de modelo propuesto consiste en agrupar usuarios en clusters y entonces predecir el rating del usuario sobre un 'item usando los ratings sobre los usuarios del mismo cluster. En vez de vecinos cercanos, clusters.
\item Tambi'en se han propuesto modelos Bayesianos para modelar dependencias entre 'items.
\item Reglas de asociaci'on.
\item Algoritmos de cluster probabilistas para permitir a usuarios pertenecer a distintos grupos.
\item Jerarqu'ias de clusters, as'i, si un cluster dado no tiene una opini'on en un 'item particular, se puede considerar al super-cluster.
\end{itemize}

%%%%%%%%%%%%%%%%%%%%%%%%%%%%%%%%%%%%%%%%%%%%%%%%%%%%%%%%
\section{Filtrado h'ibrido}

De acuerdo a Lathia~\cite{recsys:nlathia}, los m'etodos h'ibridos combinan las t'ecnicas del filtrado basado en memoria y filtrado basado en modelo para aprovechar las ventajas de ambos y reducir las desventajas al funcionar solos. Algunos ejemplos son los siguientes:

\begin{itemize}
\item Rashid et al (2006) propuso un algoritmo para conjuntos de datos grandes que combina un algoritmo de clustering con vecinos-cercanos. La idea es agrupar a los usuarios en clusters para reducir el alto costo de medir la similaridad entre todos los pares de la comunidad y luego aplicar la t'ecnica de vecinos-cercanos para hacer las predicciones. Otro ejemplo donde usan clustering es el sistema Yoda, dise\~{n}ado por Shahabi et al(2001).
\item Cayzer y Aickelin (2002) hicieron paralelismo entre el filtrado de informaci'on y la operaci'on del sistema inmune humano para construir un filtrado novedoso.
\item El razonamiento basado en casos fu'e usado exitosamente por Caccigalupo y Plaza (2007) en el dominio de recomendar un orden coherente de canciones.
\end{itemize}

Aunque estos enfoques mejoran con respecto a las t'ecnicas aisladas, tambi'en tienen sus limitaciones.

\medskip
De acuerdo a Candillier~\cite{start:candillier09}:\\
\begin{itemize}
\item La forma m'as directa de dise\~{n}ar un sistema h'ibrido consiste en correr de forma independiente un sistema colaborativo y uno basado en contenido y luego combinar las predicciones usando un esquema de votaci'on.
\item En Balabanovic y Shoham (1997, la combinaci'on se hace forzando a los 'items a ser cercanos al perfil del usuario y altamente calificados por sus vecinos.
\item En Pazzani (1999) los usuarios son comparados de acuerdo al contenido del perfil y las medidas de similaridas son usadas en un sistema de filtrado colaborativo.
\item En Polcicova et al. (2000);Melville et al. (2002), la matriz de ratings se enriquece con predicciones basadas en contenido y luego se corre un filtro colaborativo.
\item En Vozalis y Margaritis (2004) la similaridad entre 'items se calcula usando descripciones de contenido as'i como sus vectores de ratings asociados. Se usa un filtro colaborativo basado en 'item. Los autores usan tambi'en datos demogr'aficos de los usuarios. Dos usuarios podr'ian ser considerados parecidos no solo si califican parecido a los mismos 'items sino tambi'en si pertenecen al mismo segmento demogr'afico.
\item En Han y Karypis, (2005), se propone extender la lista de predicciones de un filtro colaborativo a 'items cuyo contenido es cercano a los 'items recomendados.
\item En Wang et al. (2006) se usa la similaridad basada en contenido entre 'items para comparar usuarios no solo por sus apreciaciones comunes sino considerando las apreciaciones compartidas para 'items de contenido parecido.
\item Otra estrategia es usar un filtrado basado en contenido y usar los datos producidos del filtrado colaborativo para enriquecer las descripciones de similaridad de los 'items. Los 'items son recomendados de acuerdo a los que al usuario le gustan pero no a los que no le gustan. Las similaridades entre atributos de g'enero, nacionalidad y lenguaje se definen a mano, mientras que la medida wCosine se usa para calcular la similaridad del director y actor. Cada film contiene un atributo de identificaci'on 'unico. El identificador es comparado a los films que el usuario ha anotado previamente. La noci'on de similaridad para atributos de este tipo est'a incorporado en los algoritmos de filtrado colaborativo.
\end{itemize}

%Las t'ecnicas de filtrado colaborativo han sido implementadas m'as frecuentemente que las otras y han presentado mejores resultados.


%%%%%%%%%%%%%%%%%%%%%%%%%%%%%%%%%%%%%%%%%%%%%%%%%%%%%%%%
\section{Problemas}
De acuerdo a Lathia~\cite{recsys:nlathia}, hay tres principales tipos de problemas: relativos al algoritmo, relativos al usuario y relativos al sistema. A continuaci'on se describe en qu'e consisten.

\begin{enumerate}
\item \textbf{Relativos al algoritmo}

\begin{itemize}
\item Datos faltantes. Los perfiles del usuario suelen tener datos faltantes lo que hace que al comparar, los resultados sean imprecisos. Soluciones propuestas: t'ecnicas de reducci'on de dimensionalidad Paterek (2007) y algoritmos de predicci'on de datos faltantes Ma et al (2007).
\end{itemize}

\item \textbf{Relativos al usuario}
\begin{itemize}
\item Problema del \textit{Cold Start} ocurre cuando un nuevo usuario no ha dado ratings o un nuevo 'item no ha recibido aun ratings. El sistema carece de datos para generar recomendaciones apropiadas. Estos problemas pueden abordars usando enfoques h'ibridos. Otras soluciones: en el caso de sistemas basados en ratings expl'icitos el sistema puede solicitar a los usuarios calificar cierto n'umero de 'items como parte del procedimiento de registro.
\item Efecto que las recomendaciones tienen sobre el usuario. Los usuarios buscan informaci'on nueva e interesante y si el sistema no da recomendaciones que no le aporten nada nuevo no le interesar'a. Esto es un problema abierto.
\item Los sistemas de recomendaci'on requieren tiempo para actualizarse y los usuarios tienen que esperar para que las recomendaciones cambien.
\end{itemize}

\item \textbf{Relativos al sistema}
\begin{itemize}
\item Vulnerabilidades del sistema. Se refieren al conjunto de problemas causados por usuarios maliciosos que intentan modificar el sistema. Ejemplos: creaci'on de perfiles falsos para influenciar las recomendaciones del sistema. A estos ataques se les conoce como complicidad (shilling), inyecci'on de perfil o ataques Sybil.
\item Los ataques por inyecci'on de perfil a menudo se clasifican de acuerdo a la cantidad de informaci'on que el ataque requiere para construir perfiles falsos.
\item La investigaci'on en el 'area de vulnerabilidades se divide en dos categor'ias: por un lado, los administradores del sistema requieren medios de identificar ataques, reconociendo cu'ando ocurren y cu'ales son los perfiles maliciosos. Un perfil malicioso puede identificarse si comparte una alta similaridad con un gran subconjunto de usuarios, si tiene un efecto fuerte en la precisi'on predictiva del sistema e incluye ratings que tienen una fuerte desviaci'on de la media entre los miembros de la comunidad. El problema es que un perfil de un usuario honesto puede identificarse como malicioso.
\item Por otra parte, est'a la prevenci'on de los ataques. \textquestiondown C'omo puede minimizarse el costo o efecto de los ataques? Las soluciones generales involucran minimizar el n'umero de recomendadores con quienes los usuarios pueden interactuar imitando el comportamiento social de personas desconocidas y no confiables.
\end{itemize}
\end{enumerate}

%%%%%%%%%%%%%%%%%%%%%%%%%%%%%%%%%%%%%%%%%%%%%%%%%%%%%%%%
\section{Tendencias}
Seg'un Lathia~\cite{recsys:nlathia}\\
\begin{itemize}
\item Apoyo al 'area por Netflix, cuya competencia busca reducir el error en las recomendaciones de pel'iculas Netflix.
\item La investigaci'on actual ha ignorado el aspecto temporal de los sistemas de recomendaci'on. La 'unica excepci'on es la definici'on del problema cold-start. El aspecto temporal dar'a informaci'on sobre c'omo crece, evoluciona en el tiempo y la influencia que la variaci'on de ratings disponibles tiene en la precisi'on del sistema. Es decir, dar'a claridad en el efecto de los algoritmos de filtrado en la comunidad de usuarios. Un m'etodo es considerar el sistema de recomendaci'on como un grafo.
\item La mayor'ia de la investigaci'on ha sido sobre contextos espec'ificos. Las recomendaciones con contextos cruzados sigue siendo una pregunta abierta. Dado un perfil de usuario con preferencias de pel'iculas, \textquestiondown se le puede recomendar m'usica de forma exitosa?
\item Los usuarios tienden a hacer diversos perfiles en distintas ubicaciones. Encontrar medios para portar perfiles de un lugar a otro y usarlos para recomendaciones de contexto cruzado es tambi'en un aspecto a explorar.
\item Filtrado colaborativo en ambientes m'oviles. Al usuario le permitiri'ia recibir recomendaciones dependiendo del lugar en el que est'e. Algunos aspectos a considerar son: \textquestiondown d'onde se almacenar'ian los perfiles y con qui'enes se compartir'ian?, \textquestiondown c'omo se obtendr'ian las recomendaciones?. Importancia de la privacidad y seguridad de los datos.
\end{itemize}

Seg'un Candillier~\cite{start:candillier09}\\
\begin{itemize}
\item Combinaci'on de diferentes sistemas de recomendaci'on.
\item Aprendizaje de m'etricas de similaridad.
\item Evoluci'on temporal de los ratings donde los m'as recientes cuentan m'as que los antiguos.
\item No tomar en cuenta s'olo la precisi'on de los mejores algoritmos. Es 'util tambi'en considerar aspectos como calidad y utilidad (e.g. cobertura, complejidad algor'itmica, escalabilidad, novedad, confianza y satisfacci'on de usuario).
\end{itemize}


%%%%%%%%%%%%%%%%%%%%%%%%%%%%%%%%%%%%%%%%%%%%%%%%%%%%%%%%
\section{Prototipo}
En esta secci'on se describen los aspectos considerados para el prototipo del sistema de recomendaci'on realizado por Rafael Ponce. La informaci'on fu'e obtenida los reportes~\cite{rep1:isra,rep2:isra} que muestran el avance en la tesis doctoral: B'usquedas contextuales de servicios basados en localizaci'on en un entorno de web social.


\subsection{Motivaci'on}
Una de las principales motivaciones es la dificultad del usuario para realizar una petici'on precisa en la b'usqueda de servicios, ya sea por el gran n'umero de resultados o por la imprecisi'on de dichos resultados (irrelevancia). La creciente relevancia de la informaci'on geogr'afica lo que se refleja en las numerosas APIs y aplicaciones para creaci'on de mashups y servicios afines. Es importante aprovechar la informaci'on y localizaci'on de lugares y puntos de inter'es de los usuarios. Ayudar a la b'usqueda en servicios geo-localizables considerando la individualidad del usuario.

\textbf{Aspectos importantes}: tomar en cuenta las caracter'isticas individuales de los usuarios como por ejemplo, atributos sociales, culturales y econ'omicos. S'olo unas cuantas aplicaciones usan contexto.

\textbf{Idea principal}: incorporar informaci'on contextual al sistema de recomendaci'on puede mejorar la precisi'on dentro de la recuperaci'on de informaci'on de los servicios geo-localizables. Se utiliza tecnolog'ia de Web sem'antica, interacci'on con elementos de Web social, espec'ificamente, sistemas de anotaci'on social para el filtrado colaborativo de los resultados.

\begin{itemize}
\item \textbf{Enfoque}. recomendaci'on colaborativa basada en el usuario y en el 'item y una modificaci'on de ambas tomando en cuenta anotaciones sociales (Web social). Fusi'on usuario-'item.

 Se utilizan las etiquetas dadas por un usuario (anotaciones sociales) para encontrar los lugares m'as populares y recomendados por la comunidad de usuarios.
\item \textbf{Incorporaci'on de tecnolog'ias de Web sem'antica}. Web sem'antica para manejo de informaci'on contextual. Uso de sistema de mapas Web. El contexto de los perfiles de usuario y datos de los proveedores de alg'un servicio geo-localizable se representan mediante ontolog'ias.

Beneficios del manejo de ontolog'ias: se presenta una manera interoperable de almacenar y consultar informaci'on.

Realizaci'on de consultas sobre restaurantes a partir de datos del repositorio Chefmoz.
\item \textbf{Consultas espaciales} sobre base de datos geogr'afica manejados dentro de una ontolog'ia OWL.
\item \textbf{Aplicaci'on de reglas de Web sem'antica} en SWRL para enlazar informaci'on contextual tanto de servicios (contexto de datos), del usuario (contexto del perfil personal) y del lugar y momento de la consulta (contexto del entorno).

Beneficios de las reglas sem'anticas: ayudan a determinar qu'e lugares son los m'as pertinentes, aprovechando la informaci'on de las ontolog'ias mencionadas.
\end{itemize}


\subsection{Recomendaci'on con filtrado colaborativo usando anotaciones sociales}

\begin{itemize}
\item \textbf{Objetivo}: implementar un sistema de recomendaci'on que considerara anotaciones sociales para generar recomendaciones de 'items.
\item Se tom'o como base el trabajo de predicci'on de 'items presentado por \linebreak Tso-Sutter~\cite{tagaware:tso}.
\item Se utiliz'o inicialmente un repositorio de MovieLens: datos sobre usuarios, pel'iculas, ratings y anotaciones sociales.
\item Se implement'o el enfoque de vecinos cercanos.
\item Se fusionaron los recomendadores para usuario e 'item.
\item Se extiende la relaci'on bidimensional $<$ 'item, atributo $>$ a tres problemas bidimensionales: $<$ usuario, anotaci'on,$>$, $<$'item, anotaci'on$>$ y $<$usuario,''item$>$. De esta manera, las anotaciones de los usuarios son vistas como 'items en la matriz usuario-'item y las anotaciones y las anotaciones a los 'items son vistas como usuarios.
\item La ponderaci'on de la similitud se efectu'o usando el Coeficiente de Correlaci'on de Pearson.
\item Se consider'o tambi'en un ponderado booleano, combinaciones propias del autor y el coeficiente de Tanimoto, que mide el traslape entre dos vectores con respecto a los elementos que comparten (Segaran 2007).
\item Para la fusi'on del filtrado colaborativo de usuario y de 'item se requiri'o normalizar ya que tienen distintas unidades. El filtrado de usuario se basa en frecuencia, el de 'item se basa en la similitud de los 'items.
\item Esta primera parte se desarroll'o en JSP e incluye:
\begin{itemize}
\item Filtrado colaborativo basado en el usuario.
\item Filtrado colaborativo basado en 'item.
\item Filtrado colaborativo basado en la fusi'on de usuario e 'item.
\item Filtrado colaborativo basado en usuario con extensi'on de anotaciones sociales.
\item Filtrado colaborativo basado en la fusi'on de usuario e 'item, cada uno con anotaciones sociales.
\end{itemize}
\item \textbf{Resultado}: con la fusi'on usuario-'item-tag se obtuvieron mejores resultados que con los enfoques sin tags.
\end{itemize}



\subsection{Ontolog'ias con informaci'on contextual para la ejecuci'on de reglas sem'anticas}

Esta etapa consiste en el uso de ontolog'ias para el manejo de la informaci'on contextual del perfil de usuario, de los proveedores de datos e informaci'on del entorno. El objetivo es encontrar un conjunto de resultados sobre servicios geo-localizables m'as apropiados a la situaci'on espacio-temporal del usuario que los solicite. Se utilizar'an reglas sem'anticas para identificar los resultados de mayor relevancia contextual.

\begin{enumerate}
\item \textbf{Definici'on del perfil de usuario usando un vocabulario ontol'ogico}. Se extendi'o el vocabulario FOAF para expresar expl'icitamente las preferencias de los usuarios.
\item \textbf{Creaci'on de una ontolog'ia con servicios geo-localizables}. Se genera una ontolog'ia con los servicios geo-localizables m'as cercanos al usuario. La latitud y longitud de su ubicaci'n se toman de la definici'on FOAF del primer punto. Esta parte es un primer filtro y reduce el n'umero de lugares y por lo tanto, la informaci'on a procesar.
\item \textbf{Obtenci'on de informaci'on contextual del entorno}. Se obtiene el clima y la hora de un lugar utilizando servicios Web para cierta latitud y longitud.
\item \textbf{Selecci'on y ejecuci'on de reglas sem'anticas}. De los pasos anteriores, se tiene informaci'on contextual del perfil de usuario, de los proveedores de datos y del entorno. Con base en esta informaci'on, se eligen las reglas sem'anticas a utilizar sobre los datos de los proveedores de servicios para identificar los restaurantes m'as adecuados.
\end{enumerate}

Detalle:
\begin{itemize}
\item \textbf{XFOAF}. Vocabulario extendido para incluir caracter'isticas contextuales del perfil de los usuarios. Usando FOAF se definen algunas caracter'isticas generales, el vocabulario Geo para la latitud y longitud del usuario y XFOAF para definir sus preferencias personales.
\item \textbf{Informaci'on contextual de los proveedores de datos}. A partir de las coordenadas geogr'aficas del usuario se obtiene mediante una consulta SQL los servicios geo-localizables mas cercanos y se ejecutan sobre ellos el proceso de inferencia con reglas sem'anticas. Se crea una ontolog'ia tomando los lugares geo-localizables cercanos indicados por la base de datos espacial y como vocabulario una ontolog'ia para restaurantes desarrollada previamente. Cada lugar es un ejemplar de la ontolog'ia y los campos de la base de datos son propiedades de objetos y tipo de datos. La ontolog'ia con ejemplares se considera una declaraci'on de relaciones de tipo booleano con respecto a los atributos que presentan los servicios geo-localizables. Estas relaciones se usan para determinar si el valor que presentan estos servicios (restaurantes) coinciden con los valores descritos por el perfil de usuario.
\item \textbf{Obtenci'on de informaci'on contextual del entorno}. Se usaron dos servicios Web, uno para obtener el clima y otro para obtener la hora de un determinado lugar geogr'afico.
\item \textbf{Reglas sem'anticas}. Las reglas se aplican sobre: caracter'isticas del perfil de usuario, datos de los proveedores e informaci'on contextual. Dependiendo de una caracter'istica espacio-temporal y su valor, se especifica un antecedente para esa situaci'on, con su respectivo consecuente. Se definen relaciones booleanas cuyos atributos son instanciados con los valores extra'idos de los proveedores. Se contabilizan para cada servicio, las caracter'isticas contextuales que cumple el servicio de acuerdo al perfil contextual del usuario y el entorno. La cuenta es la base para calcular el factor de impacto contextual y el ranqueo de los resultados a mostrar al usuario. Para la contabilizaci'on se usa el lenguaje de consulta SQWRL.
\end{itemize}

%%%%%%%%%%%%%%%%%%%%%%%%%%%%%%%%%%%%%%%%%%%%%%%%%%%%%%%%
\section{Estado actual del trabajo (2010)}

\begin{itemize}
\item Se pensaba que un enfoque que considere la fusion entre los sistemas de recomendaci'un basado en usuario y en 'item con extension de anotaciones sociales pod'ia mejorar el desempeño en comparaci'on a los que no usan anotaciones.
\item La mejora obtenida a trav'es de la fusi'on es casi imperceptible, mostrando puede ser suficiente considerar el acercamiento basado en usuario solamente, con la extension de anotaciones.
\item Se sugiere un conjunto de reglas, basado en estudios de marketing.
\item Las reglas de comportamiento del usuario se formalizaron en el lenguaje de SWRL.
\item Las reglas contemplan los siguientes aspectos contextuales: econ'omico, psicol'ogico y motivacional.
\item No toda la informaci'on puede ser representada por una regla.
\end{itemize}

La metodolog'ia para la construcci'on de reglas de Web sem'antica bajo un enfoque de marketing es la siguiente:

\begin{enumerate}
\item \textbf{Determinaci'on del mercado objetivo}. Consiste en definir qu'e tipo de productos o servicios se van a recomendar y el alcance del mercado (global o local). En este caso, servicios de restaurantes a nivel local.
\item \textbf{Estudio de mercado}. Consiste en informaci'un proveniente de distintas fuentes, estudios de comportamiento, psicol'ugicos, sociales y culturales, para comprender el contexto en el que se desenvuelve el mercado.
\item \textbf{Determinaci'on de patrones}. Se sigui'o literatura disponible de marketing, misma que permiti'o identificar el conjunto de reglas que se elaboraron para esta investigaci'on.
\item \textbf{Desarrollo de una ontolog'ia}. Existen numerosas metodolog'ias para la creaci'on de ontolog'ias, como la de Uschold \& King. Este paso permite asentar y formalizar los conceptos que ser'an usados en la generaci'on de reglas. Esta etapa tambi'en incluye la poblaci'on o un mecanismo que permita poblar a la ontolog'ia, con
informaci'on de los clientes potenciales y sus atributos as'i como de la oferta disponible. En nuestro caso de estudio, se identifica a un usuario, as'i como sus atributos (especificados en una versi'on extendida bajo FOAF), la ontolog'ia
se puebla con ejemplares de restaurantes cercanos a la localizaci'un del usuario, as'i como los atributos de la misma.

\item \textbf{Formalizaci'on de reglas}. Es el nexo entre los patrones identificados y la ontolog'ia desarrollada. Las reglas de Web sem'antica formuladas podr'an hacer uso de la ontolog'ia e inferir los resultados que cumplan las premisas especificadas. En nuestro caso, el lenguaje para la especificaci'on de las reglas es SWRL.
\item \textbf{Pruebas y correcciones}.
\item \textbf{Adaptaci'on evolutiva de las reglas de acuerdo a cambios del entorno}. Modificaci'on, eliminaci'on o integraci'on de reglas.
\item \textbf{Ranking con base a reglas de Web sem'antica}. Rankeo basado en la cantidad de aspectos contextuales cumplidos entre los contextos de perfiles: usuario/datos, usuario/entorno, datos/entorno. Se desarroll'o un algoritmo que contabiliza las reglas cumplidas de acuerdo a la uni'on de contextos correspondientes. Al algoritmo se le agrega la combinaci'on de la matriz con los resultados de la recomendaci'on social para que en la experimentaci'on se pruebe la fusi'on de ambas partes, la contextual y la social. Se considera un peso de $0.1$ a $0,9$ en variaciones de $0.1$ donde se pruebe el desempe\~{n}o de las recomendaciones y determinar si se le debe dar mayor peso a uno o al otro.
\end{enumerate}
%%%%%%%%%%%%%%%%%%%%%%%%%%%%%%%%%%%%%%%%%%%%%%%%%%%%%%%%
\section{Evaluaci'on de sistemas de recomendaci'on}

Tesis~\cite{eemcs13083},Lathia pag 11,3301 pag 8,recsyssurvey2010 pag 5, secci'on 3,\cite{Herlocker04evaluatingcollaborative}\\

El filtrado colaborativo ha sido usando extensivamente lo que ha originado una gran diversidad de algoritmos para generar recomendaciones. Como en otras 'areas, todos dicen ser el mejor; sin embargo, no existe una metodolog'ia de evaluaci'on estándard para determinar qu'e hace a un algoritmo mejor que otro.

La evaluaci'on de los sistemas de recomendaci'on es dif'icil por diversas razones:
\begin{enumerate}
\item Muchos algoritmos se han dise\~{n}ado para conjuntos de datos espec'ificos.
\item Los objetivos de la evaluaci'on pueden ser diferentes. Muchos se enfocan a medir la precisi'on de sus predicciones de ratings. Otros, miden la frecuencia con que el sistema llev a tomar malas decisiones. Otros miden la cobertura de 'items del algoritmo. Tambi'en se ha medido la habilidad del sistema de explicar las recomendaciones.
\item Algunos investigadores afirman que la medida m'as 'util es la satisfacci'on del usuario. Los sistemas comerciales miden la satisfacci'on del usuario por el la cantidad de productos comprados y no devueltos mientras que los no comerciales preguntan a los usuarios qu'e tan satisfechos est'an.
\item Se observa que las mejoras en sistemas de filtrado colaborativo pueden ser en distintas direcciones. La evaluaci'on puede ser qu'e tan bien comunican su razonamiento a los usuarios, obtener recomendaciones precisas con el m'inimo de datos. Considerando esto, se necesitar'an nuevas m'etricas.
\end{enumerate}



%-------------------------------------------------------
\subsection{Definir la tarea del sistema}
La adecuada evaluaci'on de un sistema de recomendaci'on depende de los objetivos y las tareas para los que el sistema fu'e hecho. Las siguientes son las tareas m'as evaluadas en la literatura
\begin{itemize}
\item Anotaci'on en contexto: Filtrado de contenido. Predicci'on de preferencia de contenido. El usuario selecciona los mensajes o ligas a seguir de una lista que se le muestra. La lista se cre'o por predicciones. El factor a evaluar es qu'e tan bien el sistema ayud'o al usuario a distinguir entre contenido deseado y no deseado.
\item Encontrar buenos art'iculos. El sistema sugiere una lista rankeada de art'iculos al usuario y la predicci'on de qu'e tanto le gustar'a. Los comerciales no muestran la predicci'on.
\end{itemize}

Las siguientes son tareas consideradas importantes pero que hasta el 2004 no se hab'ian tomado en cuenta para evaluaci'on.

\begin{itemize}
\item Encontrar todos los items buenos. Algunos usuarios buscan encontrar todos los 'items de su inter'es sin importar el tiempo y sin importar ver unos cuantos que no les interesen. Les importa que el sistema les de garant'ia de que el n'umero de falsos positivos sea m'inimo. Factor: la cobertura.
\item Secuencia recomendada. Se asume que la secuencia de art'iculos, como un todo, es la que se busca para satisfacci'on. Como ejemplo, la m'usica y los art'iculos.
\item Sin comprar, s'olo navegar. Los recomendadores se enfocan a evaluar la precisi'on para ayudar al usuario a decidir una compra. No siempre es as'i, por lo que no necesariamente la precisi'on es la forma correcta de evaluar. Aspectos: facilidad de navegaci'on y el nivel y naturaleza de la informaci'on que se le de al usuario.
\item Encontrar un recomendador confiable. Aspecto: evaluar si los usuarios contribuir'an a los ratings pues eso es lo que hace que el recomendador sea exitoso.
\item Mejorar el perfil. Es la tarea relacionada con el rating que la mayor'ia de los sistemas de recomendaci'on suponen. Los usuarios contribuyen a los ratings porque creen que mejoran su perfil y as'i mejoran la calidad de la recomendaci'on que reciben.
\item Deseo de expresarse. Hay usuarios a quienes no les interesan las recomendaciones sino expresar su opini'on. Aspecto: favorecer la autoexpresi'on puede proporcionar m'as datos para mejorar las recomendaciones.
\item Ayudar a otros. Relacionado con el anterior pero no necesariamente van juntos.
\item Influenciar a otros. Totalmente mala intenci'on, hay usuarios que tratan de influenciar a otros mostrando ciertos art'iculos, e.g., pel'iculas por estrenar. Aspecto: evaluar qu'e tanto el sistema previene esta tarea.
\end{itemize}
En el 'area de Interacci'on Humano Computadora (HCI) se dice que el proceso de evaluaci'on debe empezar por entender las tareas que el sistema facilitar'a. Si se eval'ua al sistema por el beneficio al usuario se empezar'ia por identificar la tarea principal.
%-------------------------------------------------------
\subsection{Conjuntos de datos}
Existen distintos aspectos que pueden hacer que los conjuntos de datos influyan en el 'exito del sistema de recomendaci'on.

\bigskip
\noindent
\textbf{Offline/Online}.
La evaluaci'on puede realizarse offline o se necesitan datos en vivo? Si se eval'ua el sistema en la tarea: encontrar buenos 'items donde se desea recomendar 'items novedosos puede no convenir usar solamente datos offline. Es probable que el conjunto de datos no proporcione suficiente informaci'on para evaluar la calidad de los 'items que se est'an recomendando. Si hay un 'item totalmente desconocido para el usuario es probable que no haya rating de ese usuario en esa base de datos. En vivo, los ratings se pueden obtener en el acto por cada item recomendado. Se podr'ia combinar. Predomina el la evaluaci'on offline de precisi'on predictiva. Ventaja: es r'apida y econ'omica. Desventajas: escasez de datos, se limita a evaluaci'on objetiva (datos cuantitativos, los cualitativos no se puede); no se pueden saber preferencias de uso del sistema, por ejemplo. En vivo se puede medir desempe\~{n}o, satisfacci'on, participaci'on entre otras medidas.

\bigskip
\noindent
\textbf{Uso de datos simulados}
Si no hay datos disponibles \textquestiondown pueden usarse datos simulados? Si se usando un nuevo dominio conviene usar datos sint'eticos. Oportunidad: t'ecnicas de modelado para generar datos sint'eticos de acuerdo a un modelo para evaluar el algoritmo. Esto llevar'ia a recomendadores m'as precisos con propiedades teo'ricas m'as claras.

\bigskip
\noindent
\textbf{Propiedades de los datos}
\textquestiondown qu'e propiedades debe tener para modelar mejor las tareas que se van a evaluar del sistema? Cuando se dise\~{n}a un sistema que recomienda comandos, se espera que el usuario haya usado del 5 al 10\%. No se puede usar los mismos algoritmos que para sugerir art'iculos bas'andonos en la evaluaci'on de resultados. Herlocker las divide en tres categor'ias:
	\begin{enumerate}
	\item \textbf{Atributos de dominio}. Reflejan la naturaleza del dominio que se va a recomendar. Puede incluir:
	\begin{itemize}
	\item el t'opico a recomendar y su contexto asociado;
	\item las tareas que el sistema soporta;
	\item la necesidad de calidad y novedad;
	\item el costo-beneficio asociado a una buena o mala recomendaci'on y
	\item la granularidad de las preferencias del usuario.
	\end{itemize}
	\item \textbf{Atributos de inherencia}. Refleja la naturaleza del sistema de recomendaci'on espec'ifico para el que fueron hechos los datos. Incluye atributos sobre los ratings:
	\begin{itemize}
	\item Si son expl'icitos, impl'icitos o ambos.
	\item Escala del rating.
	\item Dimensiones del rating (e.g., pel'icula, actores, director).
	\item Si se registran las recomendaciones.
	\item Disponibilidad de informaci'on demogr'afica del usuario (e.g., zona postal para recomendaci'on de un lugar).
	\item Sesgo de la colecci'on de datos: si se le pide a usuarios que califiquen ciertos 'items, los resultados son distintos.
	\end{itemize}
	\item \textbf{Atributos de muestreo}. Refleja la distribuci'on de los datos. Son las propiedades estad'isticas de los datos:
	\begin{itemize}
	\item Densidad de los ratings: porcentaje promedio de 'items que han sido calificados por usuario; la densidad puede ser manipulada incluyendo o agregando 'items.
	\item Densidad de ratings de los usuarios para cuyas recomendaciones se hacen lo que representa la experiencia del usuario en el sistema al momento de la recomendaci'on; ratings de usuarios con experiencia significativa puede ocultarse para simular que son nuevos usuarios.
	\item Tama\~{n}o general y distribuci'on del conjunto de datos; algunos conjuntos tienen m'as usuarios que 'items.
	\end{itemize}
	\end{enumerate}

\bigskip
\noindent
\textbf{Escasez de datos}. Existen pocos conjuntos de datos de prueba disponibles, adem'as con poca diversidad. Esto origina que por la falta de datos se restrinja el tipo de experimentos e hip'otesis.
%-------------------------------------------------------
\subsection{M'etricas de precisi'on}

En la mayor'ia de los sistemas de recomendaci'on los algoritmos son evaluados por su precisi'on. Una m'etrica de precisi'on mide emp'iricamente qu'e tan cerca la predicci'on del ranking de 'items para un usuario difiere de la preferencia real del usuario. Los ratings estimados son comparados contra el rating actual~\cite{start:candillier09}. Tambi'en miden qu'e tan bien un sistema puede predecir un rating exacto para un 'item espec'ifico\cite{Herlocker04evaluatingcollaborative}.

Seleccionar una m'etrica adecuada es un reto debido a la diversidad de m'etricas publicadas. Sin un est'andar, los investigadores contin'uan introduciendo nuevas m'etricas al evaluar sus sistemas lo que origina que sea m'as dif'icil la comparaci'on. Las m'etricas de precisi'on pueden clasificarse en tres clases: 

\begin{enumerate}
\item \textbf{Precisi'on predictiva}. Miden qu'e tan cerca están las predicciones de los ratings del sistema de recomendaci'on a los del usuario real. Son particularmente importantes para evaluar tareas en las que el rating calculado se desplegar'a al usuario. Como los valores calculados generan un orden, tambi'en se usa para medir la habilidad del sistema para rankear los 'items seg'un las preferencias del usuario. La desventaja es que la m'etrica se limita a calcular la diferencia entre el rating estimado y el verdadero.

	\begin{itemize}
	\item Error absoluto medio (MAE). Mide la desviaci'on del promedio absoluto entre un rating estimado y el rating verdadero. No apropiado para \textit{Encontrar buenos 'items} donde se despliega una lista rankeada pero el usuario solamente ve los de mayor ranking. Puede no ser importante que tan exactas son las predicciones para 'items que el sistema sabe que al usuario no le interesan. Tampoco es muy apropiada cuando la granularidad del la preferencia es peque\~{n}a ya que los errores s'olo afectar'an a la tarea si el resultado es clasificado err'oneamente como bueno o malo. Un error en el cual el sistema estima un 4 en vez de un 5 no hace una diferencia al usuario. Tres medidas relacionadas con el error absoluto medio son: el error cuadr'atico medio, la ra'iz cuadrada del error cuadr'atico medio y el error absoluto de la media normalizada.
	\item Error cuadr'atico medio.
	\item Ra'iz cuadrada del error cuadr'atico medio (RMSE). 
	MAE y RMSE se enfocan en la diferencia entre el rating de un 'item $i$ por un usuario $a$ y la predicci'on para el mismo usuario y el 'item. En general, ambas m'etricas miden lo mismo y se comportan de forma parecida. Si en un experimento se obtiene una MAE reducida, el RMSE tambi'en se reducir¿a. La diferencia radica en el grado en el cual los errores son penalizados.
	El error cuadr'atico medio y RMSE elevan al cuadrado el error antes de sumarlo lo que da mayor 'enfasis en errores grandes. Por ejemplo, un eeror de un punto incrementea la suma del error en uno, pero el error de dos puntos incrementa la suma en cuatro. 
	\item Error absoluto de la media normalizada.  Es el error absoluto de la media normalizada con respecto al rango de valores de ratings. En teor'ia permite la comparaci'on entre predicciones sobre diferentes conjuntos aunque no se ha investigado.
	\end{itemize}

Un error en la predicci'on afecta el error medio de la misma manera, sin importar si la predicci'on permite a la entrada calificar como recomendaci'on top o no. Adem'as, muchos 'items tendr'an baja varianza. Como consecuencia de esto, un m'etodo de evaluaci'on que s'olo hace predicciones sobre 'items en el conjutno de prueba, 'items que el usuario ha calificado, tender'a a mostrar buen desempe\~{n}o. Los sistemas reales, que tienen que hacer predicciones en todos los 'items no calificados tienen peor desempe\~{n}o. 

Los errores de la media no tienden a reflejar la experiencia con el usuario final. Estas m'etricas permanecen debido a la necesidad de evaluaciones emp'iricas que puedan comparar el desempe\~{n}o relativo para diferentes t'ecnicas sin incluir puntos de vista subjetivos. 

\item \textbf{Precisi'on de clasificaci'on}. Miden la frecuencia con la que el sistema toma decisiones correctas o incorrectas sobre si un 'item es bueno. Son apropiadas para tareas como \textit{encontrar buenos 'items} donde los usuarios tienen preferencias binarias.

En experimentos offline, las m'etricas de clasificaci'on pueden tener problemas de escasez de datos. El problema ocurre cuando el sistema a ser evaluado genera una lista de los 'items m'as recomendados. Cuando se eval'ua la calidad de la lista, las recomendaciones puede ser que no hayan sido calificadas. Lo que se haga con esos 'items puede llevar a ciertos sesgos.

Un enfoque con datos escasos es ignorar las recomendaciones de 'items para los que no hay ratings. De la lista de recomendaciones se remueven los 'items no calificados. La tarea de recomendaci'on se ha cambiado a una de \textit{predecir los 'items m'as recomendados que no han sido calificados}. En tareas donde el usuario s'olo observa las mejores recomendaciones esto puede llevar a evaluaciones no precisas con respecto a la tarea del usuario. El problema es que la calidad de los 'items que el usuario actualmente ver'ia no puede ser medida. Por ejemplo, si hay un 'item que ha sido calificado por un s'olo usuario con la mayor calificaci'on, 'ese 'item aparecer'a en la lista de los mejores aun cuando s'olo un usuario lo ha calificado. Sin embargo, como ning'un otro usuario lo ha evaluado, la recomendaci'on ser'a ignorada por la métrica de evaluaci'on, salt'andose la falta en el algoritmo.

Otro enfoque es suponer ratings por omisi'on, a menudo ligeramente negativos para 'items no calificados. Lo malo es que el rating por omisi'on puede ser muy distinto del rating verdadero no observado.

Un tercer enfoque es calcular qu'e tanto de los 'items m'as recomendados se encuentra en la lista de recomendaciones generada por el sistema. B'asicamente se est'a midiendo qu'e tan bien el sistema puede identificar 'items de los que el usuario est'a consciente. Este enfoque puede generar sistemas que se sesguen hacia las recomendaciones obvias, ajust'andose a datos conocidos y pobremente a nuevos datos.

Las m'etricas de clasificaci'on no tratan de medir directamentve la habilidad de un algoritmo para predecir ratings. Desviaciones de los ratings actuales son tolerados mientras no lleven a errores de clasificaci'on.


\item \textbf{Precisi'on/recuerdo y m'etricas relacionadas}. Son las m'etricas m'as usadas para evaluar sistemas de recuperaci'on de informaci'on. La precisi'on y el recuerdo se calculan en una tabla de $2 \times 2$. El conjunto de 'items es separado en dos clases: relevante o no relevante. Se necesita separar el conjunto de 'items en  el conjunto que fue regresado al usuario (seleccionado/recomendado), y el que no. Se asume que el usuario considerar'a todos los 'items regresados.

La precisi'on se define como la relaci'on de 'items seleccionados relevantes con respecto al n'umero de 'items seleccionados (Ec.~\ref{eqprec}).

\begin{equation}
P = \frac{N_{rs}}{N_s}
\label{eqprec}
\end{equation}

La precisi'on representa la probabilidad de que un 'item seleccionado sea relevante. 

El recuerdo (Ec.~\ref{eqrec}) se define como la relaci'on de 'items seleccionados relevantes con el n'umero total de 'items relevantes disponibles. El recuerdo representa la probabilidad de que un 'item relevante sea seleccionado.


\begin{equation}
R = \frac{N_{rs}}{N_r}
\label{eqrec}
\end{equation}

La precisi'on y el recuerdo dependen de la separaci'on de 'items relevantes y no relevantes.

La relevancia es un t'ermino subjetivo en los sistemas de recomendaci'on pues s'olo el usuario puede determinar si el 'item le gust'o o no. Tambi'en las escalas del rating pueden diferir al considerarlas relevantes o no. 

El recuerdo requiere saber qu'e 'item es relevante. En muchos casos, esto es inconsistente pues para cada usuario los 'items pueden ser o no relevantes.

Una forma para calcular la precisi'on y el recuerdo ser'ia predecir los $N$ mejores 'items para los que se tienen ratings. Se toman los ratings del usuario, se dividen en conjunto de entrenamiento y prueba, se entrena el algoritmo y se predice los $N$ mejores 'items para el conjunto de prueba. Si suponemos que la distribuci'on de 'items relevantes y no relevantes dentro del conjunto de prueba del usuario es la misma que la istribuci'on verdadera, entonces la precisi'on y el recuerdo tendr'a una aproximaci'on m'as cercana.

Precisi'on y recuerdo deben usarse juntos para comparar los algoritmos. Son inversamente relacionados y dependen de la longitud de la lista resultante. Cuando se devuelven m'as 'items, el recuerdo aumenta y la precisi'on disminuye.

La precisi'on aislada puede ser apropiada si el usuario no necesita una lista de todos los 'items relevantes potenciales tal como en la tarea de \textit{encntrar buenos 'items}. Si la tarea es encontrar a todos los 'items relevantes, entonces el recuerdo tambi'en es importante. 

Al igual que todas las m'etricas de clasificaci'on, precisi'on y recuerdo son menos apropiados para dominios con granularidad no binaria. Como la precisi'on y el recuerdo miden 'unicamente relevancia binaria, no pueden medir la calidad del orden entre 'items que son seleccionados como relevantes.

\item \textbf{Curvas ROC}

\end{enumerate}






%-------------------------------------------------------
Los sistemas de recomendaci'on no deben dar s'olo precisi'on sino utilidad. Un recomendador que proporciona alta precisi'on pero solo calcula predicciones para 'items f'aciles de predecir y que son los que menos necesitan ser recomendados. Un sistema que siempre recomienda 'items populares puede prometer que a los usuarios les gustar'an sus recomendaciones pero eso es algo que una simple m'etrica de popularidad tambi'en hace.

El desempe\~{n}o de un sistema de recomendaci'on debe evaluarse con respecto a tareas espec'ificas y el contexto del dominio; no solamente usando precisi'on. Algunas medidas son las siguientes:

\subsection{Cobertura}
Mide el porcentaje de un conjunto de datos sobre el cual el sistema es capaz de dar recomendaciones.

Es especialmente importante para la tarea de \textit{encontrar todos los 'items buenos} ya que los sistemas que no pueden evaluar a la mayor'ia de los 'items en el dominio no pueden encontrar todos los 'items buenos. Tambi'en es importante para la tarea de \textit{Annotate in context} ya que no pueden hacerse anotaciones para los 'items en los que no hay predicciones.

En predicci'on, la cobertura se refiere al porcentaje de 'items para el cual el sistema puede hacer predicciones; es la cobertura de predicci'on. Otro tipo de cobertura contesta a \textquestiondown qu'e porcentaje de 'items son los que el sistema recomienda siempre? Para un sitio comercial, mide qu'e tanto del cat'alogo se recomienda por lo que se le llama cobertura de cat'alogo. Se hace en combinaci'on con precisi'on.

Una alternativa es medir la cobertura sobre los 'items que el usuario tiene inter'es. No es una desventaja si un sistema no muestra cosas sobre futbol a alguien que no le interesa.

La cobertura de cat'alogo se expresa como el porcentaje de 'items que son recomendados a los usuario. Generalmente se mide en el conjunto de recomendaciones en un punto en el tiempo. Por ejemplo, puede medirse tomando la uni'on de las diez mejores recomendaciones para cada usuario. Esta m'etrica se distorsiona si no se toma en cuenta la precisi'on. Por ejemplo, si hay un 'item en el cat'alogo que no es interesante a ning'un usuario, un buen algoritmo no deber'ia recomendarlo nunca lo que bajar'ia la cobertura pero elevar'ia la precisi'on.

Una medida de cobertura debe tener las siguientes caracter'isticas: (1) deber'ia medir cobertura de predicci'on y de cat'alogo; (2) para la cobertura de predicci'on, deber'ia dar m'as peso a los 'items que probablemente le gustar'an m'as al usuario;(3) deber'ia haber algo que combine la cobertura con la precisi'on para dar una medida de 'precisi'on pr'actica'

Comparar recomendadores con estas dimensiones asegurar'a que los nuevos recomendadores no logren la precici'on por escoger 'items f'aciles de recomendar sino que den un amplio rango de recomendaciones 'utiles a los usuarios.

%-------------------------------------------------------

\subsection{Tasa de aprendizaje}
Mide qu'e tan r'apido un algoritmo puede producir buenas recomendaciones.

El desempe\~{n}o del sistema var'ia dependiendo de la cantidad de datos de aprendizaje disponible. Conforme la cantidad de datos de entrenamiento aumenta, la calidad de las recomendaciones deber'ia incrementar. Algunos algoritmos requieren pocos datos; otros necesitan muchos. Tres son las tasas de aprendizaje consideradas en los sistemas de recomendaci'on:

\begin{itemize}
\item Tasa de aprendizaje general. Es la calidad de la recomendaci'on en funci'on del n'umero general de ratings (o usuarios) en el sistema).
\item Tasa de aprendizaje por 'item. Es la calidad de las predicciones para un 'item en funci'on del n'umero de ratings disponible para cada 'item.
\item Tasa de aprendizaje por usuario. Es la calidad de las recomendaciones para un usuario en funci'on del n'umero de ratings con los que el usuario ha contribuido.
\end{itemize}

El m'etodo m'as com'un para comparar las tasas de aprendizaje de diferentes algoritmos es graficar la calidad (precisi'on) contra el n'umero de ratings.

La falta de evaluaci'on de la tasa de aprendizaje se debe a que las bases de datos de prueba tienen ya muchos ratings.

%-------------------------------------------------------
\subsection{Novedad y recomendaci'on afortunada}
Mide si una recomendaci'on es novedosa.

Algunos sistemas pueden dar recomendaciones con alta precisi'on y una cobertura razonable pero ser in'utiles. Esto se debe a que recomiendar 'items populares, estad'isticamente con alta precisi'on pero que no es necesario recomedar porque a todos les gustan (e.g., pan, pl'atanos). 

Aunque las recomendaciones obvias no dan valor agregado, a los usuarios les gusta recibir algunas recomendaciones de 'items que les son familiares. Lo que se ha hecho es incrementar la confianza en el sistema. En el caso de la m'usica o art'iculos gratuitos, los usuarios prefieren cosas novedosas. 

Dependiendo de la tarea, es mejor mostrar un gran n'umero de 'items familiares que en otras.

Adem'as de la dimensi'on de \textit{novedad} est'a la de \textit{casualidad}. Una recomendaci'on afortunada ayuda al usuario a encontrar un 'item sorprendentemente interesante que de otra manera no lo habr'ia descubierto.

La diferencia entre novedad y recomendaci'on afortunada es importante cuando se eval'uan algoritmos de filtrado colaborativo porque el potencial para recomendaciones afortunadas es una faceta del filtrado colaborativo que el filtrado basado en contenido no tiene.

Dise\~{n}ar m'etricas para medir las recomendaciones afortunadas es dif'icil porque es una medida del grado en que las recomendaciones muestran 'items que son a la vez atractivas y sorprendentes.

Una buena m'etrica debe encontrar la forma de ver c'omo los intereses del usuario se ampl'ian sobre el tiempo. Ver qu'e tan feliz es con los 'items recomendados. 

Una buena m'etrica de novedad ver'ia de forma m'as general qu'e tan bien un sistema hace al usuario consciente de 'items desconocidos. Hasta d'onde acepta nuevas recomendaciones?

%-------------------------------------------------------
\subsection{Confianza}
Ayudan al usuario a tomar decisiones efectivas.

Los usuarios de sistemas de recomendaci'on frecuentemente tienen qu'e decidir c'omo interpretar las recomendaciones en dos dimensiones conflictivas. La primera es la fuerza de la recomendaci'on, i.e., qu'e tanto el sistema cree que al usuario le gustar'a un 'item. La segunda es la confianza, i.e., qu'e tan seguro est'a el sistema de que su recomendaci'on es precisa.

Muchos suponen que es m'as probable que un usuario prefiera un 'item on cinco estrellas  que uno de cuatro, en escala de cinco. La suposici'on es a menudo falsa pues muchos productos a menudo se basan en peque\~{n}as cantidades de datos.

Para ayudar al usuario a tomar decisiones bas'andose en las recomendaciones, los sistemas deben ayudar al usuario a navegar entre la fuerza y la confianza. En algunos enfoques no presentan recomendaciones que vienen de conjuntos de datos reducidos. Quieren recomendaciones en las que los usuarios puedan confiar. Otro enfoque muestra que desplegando el nivel de confianza hace que el usuario tome mejores decisiones. 

Medir la calidad de la confianza es dif'icil porque es un atributo con varias dimensiones. Sin embargo, recomendadores que no incluyen alguna medida de confianza pueden llevar a pobres decisiones. 

Si el despliegue de confianza muestra una probabilidad cuantitativa o cualitativa de c'omo ser'a la precisi'on, la confianza puede probarse contra las recomendaciones actuales hechas a los usuarios. \textquestiondown Qu'e tan precisas son las recomendaciones hechas con alta confianza y con baja? 

Si el desplegar la confianza directamente influye en la decisi'on, medir la precisi'on de las decisiones puede medir la confianza. \textquestiondown Qu'e tanto mejora la toma de decisiones cuando se les muestra a los usuarios una medida de confianza?

%-------------------------------------------------------
\subsection{Evaluaci'on del usuario}
Miden si el usuario est'a satisfecho con el sistema y su desempe\~{n}o, es decir, si le fu'e 'util o no. 

Las m'etricas anteriores son aqu'ellas que se cree que afectan la utilidad del sistema al usuario. Pero, \textquestiondown c'omo se eval'ua directamente la reacci'on del usuario al sistema?

En esta parte, hay dos aspectos: la reacci'on del usuario al sistema y la reacci'on a la recomendaci'on. En la primera, se eval'ua con an'alisis de usabilidad, en la segunda, se eval'ua el proceso total: el usuario compr'o, visit'o el lugar y le gust'o, eso depende del objetivo del sistema qu'e es lo que defina como exitoso.

Existen diversas dimensiones para la evaluaci'on del usuario:
\begin{itemize}
\item Expl'icito/impl'icito. Consiste en preguntar expl'icitamente al usuario sobre sus reacciones ante el sistema mediante encuestas y entrevistas. La forma impl'icita consiste en obtener trazas del comportamiento y analizarlas.
\item Estudios de laboratorio/campo. Las pruebas en laboratorio, bajo condiciones controladas son buenas para probar hip'otesis espec'ificas. Las pruebas en campo pueden mostrar lo que los usuarios hacen en el contexto real.
\item Resultado/proceso. Para cualquier tarea se deben establecer m'etricas sobre lo que se define como un resultado exitoso. Desde la perspectiva del sistema, la medida principal ser'ia la precisi'on. Desde la perspectiva del usuario, las m'etricas son relativas a las tareas particulares.
\item Corto/largo periodo. Algunos aspectos no se ven en estudios de corta duraci'on. 
\end{itemize}

\textbf{Control}. Uno de los aspectos importantes es el control. Al usuario le gusta sentir que tiene el control de sus recomendaciones. Por ejemplo, poder corregir lo que no le gusta. Las interfaces para entrada expl'icita de preferencias se han dise\~{n}ado para permitir a los usuarios indicar las preferencias asl sistema.
%-------------------------------------------------------
\subsection{Satisfacci'on del usuario}


%%%%%%%%%%%%%%%%%%%%%%%%%%%%%%%%%%%%%%%%%%%%%%%%%%%%%%%%
%%%%%%%%%%%%%%%%%%%%%%%%%%%%%%%%%%%%%%%%%%%%%%%%%%%%%%%%
\newpage
\appendix
\section{Glosario}

\begin{itemize}
\item \textbf{Ontolog'ia}. Exhaustivo y riguroso esquema conceptual dentro de uno o varios dominios dados; con la finalidad de facilitar la comunicaci'on y el intercambio de informaci'on entre diferentes sistemas y entidades. Un uso común tecnol'ugico actual del concepto de ontolog'ia, en este sentido sem'antico, lo encontramos en la inteligencia artificial y la representaci'un del conocimiento. En algunas aplicaciones, se combinan varios esquemas en una estructura de facto completa de datos, que contiene todas las entidades relevantes y sus relaciones dentro del dominio. T'ipicamente, las ontolog'ias en las computadoras se relacionan estrechamente con vocabularios fijos –una ontolog'ia fundacional– con cuyos t'erminos debe ser descrito todo lo dem'as.
\item \textbf{OWL} (Ontology Web Language). Es un lenguaje de marcado para publicar y compartir datos usando ontolog'ias en la WWW. OWL tiene como objetivo facilitar un modelo de marcado construido sobre RDF (Resource Description Framework) y codificado en XML. Estas herramientas y otros componentes hacen posible el proyecto de web sem'antica.
\item \textbf{Web sem'antica}. Se basa en la idea de añadir metadatos sem'anticos y ontol'ogicos a la World Wide Web. Esas informaciones adicionales —que describen el contenido, el significado y la relaci'un de los datos— se deben proporcionar de manera formal, para que as'i sea posible evaluarlas autom'aticamente por m'aquinas de procesamiento. El objetivo es mejorar Internet ampliando la interoperabilidad entre los sistemas.
\item \textbf{SWRL (Semantic Web Rule Language)}. Es una propuesta para un lenguaje de reglas de la Web Sem'antica que combina sub-lenguajes de OWL y RML (Rule Markup Language). Las reglas son de la forma de una implicaci'on entre un antecedente (cuerpo) y un consecuente (cabeza). El significado puede leerse como: siempre que las condiciones en el antecedente se cumplan, entonces las condiciones especificadas en el consecuente tambi'en se cumplen. Ejemplo entendible por personas:
\begin{verbatim}
hasParent(?x1,?x2) ∧ hasBrother(?x2,?x3) ⇒ hasUncle(?x1,?x3)
\end{verbatim}

Ejemplo en XML:

\begin{verbatim}
<ruleml:imp>
  <ruleml:_rlab ruleml:href="#example1"/>
  <ruleml:_body>
    <swrlx:individualPropertyAtom swrlx:property="hasParent">
      <ruleml:var>x1</ruleml:var>
      <ruleml:var>x2</ruleml:var>
    </swrlx:individualPropertyAtom>
    <swrlx:individualPropertyAtom swrlx:property="hasBrother">
      <ruleml:var>x2</ruleml:var>
      <ruleml:var>x3</ruleml:var>
    </swrlx:individualPropertyAtom>
  </ruleml:_body>
  <ruleml:_head>
    <swrlx:individualPropertyAtom swrlx:property="hasUncle">
      <ruleml:var>x1</ruleml:var>
      <ruleml:var>x3</ruleml:var>
    </swrlx:individualPropertyAtom>
  </ruleml:_head>
 </ruleml:imp>

\end{verbatim}
\item \textbf{Mapas Web (Web Map Service WMS)} es un protocolo est'andar para proporcionar im'agenes de mapas georeferenciados en Internet. Los mapas son generados por un sevidor de mapas usando informaci'on de una base de datos de un sistema de informaci'on geogr'afica (GIS).
\item \textbf{Mashups}. En desarrollo Web, es una p'agina Web o aplicaci'on que usa y combina datos, presentaci'on o funcionalidad de dos o m'as fuentes para crear servicios nuevos. El t'ermino implica integraci'on f'acil y r'apida usando APIs abiertas y fuentes de datos para producir resultados enriquecidos que no necesariamente eran la raz'on original para producir los datos crudos iniciales.
\item \textbf{Geolocalizaci'on}. Se refiere al posicionamiento con el que se define la localizaci'un de un objeto espacial (representado mediante punto, vector, 'area, volumen) en un sistema de coordenadas determinado. Este proceso es utilizado frecuentemente en los Sistemas de Informaci'un Geogr'afica.
\item \textbf{FOAF (Friend of a friend)} es una ontolog'ia para describir personas, sus actividades y sus relaciones con otras personas y objetos. FOAF permite a grupos de gente describir redes sociales sin necesidad de una base de datos centralizada. FOAF es un vocabulario descriptivo que usa Resource Description Framework (RDF) y Web Ontology Language (OWL). Se pueden usar perfiles FOAF para encontrar por ejemplo, a toda la gente que vive en Europa.
\end{itemize}

%%%%%%%%%%%%%%%%%%%%%%%%%%%%%%%%%%%%%%%%%%%%%%%%%%%%%%%%
\newpage
\bibliographystyle{apalike}
\bibliography{ea}
\end{document}

%%%%%%%%%%%%%%%%%%%%%%%%%%%%%%%%%%%%%%%%%%%%%%%%%%%%%%%%
% \begin{figure}[h]
% \centering
% \includegraphics[keepaspectratio,width=2cm]{theimg/bp}
% \caption[Bletchley Park]{Bletchley Park}
% \label{fig:bp}
% \end{figure}
%
%
% \begin{figure}[h]
% \begin{center}
% \subfigure[Teletipo]{\includegraphics[width=2.5cm,keepaspectratio]{theimg/teletipo}}
% \hspace{1cm}
% \subfigure[Colossus]{\includegraphics[keepaspectratio,height=2cm]{theimg/colossus}}
% \caption{Teletipo y Colossus}
% \label{fig:teletipocolossus}
% \end{center}
% \end{figure}


