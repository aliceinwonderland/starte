% LaTeX document
\documentclass[11pt]{article}
\usepackage[spanish,activeacute]{babel}
\usepackage{graphicx}
\usepackage{epsfig}
\usepackage{subfigure}
\usepackage{url}
\usepackage[colorlinks=true]{hyperref}

\setlength{\topmargin}{-.5in}
\setlength{\textheight}{9in}
\setlength{\oddsidemargin}{.125in}
\setlength{\textwidth}{6.25in}
\renewcommand{\familydefault}{\sfdefault}


\begin{document}

\title{Evaluaci'on de niveles de satisfacci'on en sistemas de recomendaci'on: estado del arte}
\author{blancavg}

\maketitle
%%%%%%%%%%%%%%%%%%%%%%%%%%%%%%%%%%%%%%%%%%%%%%%%%%%%%%%%
\abstract{
Estoy haciendo una revisi'on del estado del arte sobre t'ecnicas de evaluaci'on de niveles de satisfacci'on en sistemas de recomendaci'on. Iniciar'e con conceptos sobre sistemas de recomendaci'on para familiarizarme con el tema y posteriormente me centrar'e en trabajos espec'ificos sobre evaluaci'on.
}
%%%%%%%%%%%%%%%%%%%%%%%%%%%%%%%%%%%%%%%%%%%%%%%%%%%%%%%%
\section{Sistemas de recomendaci'on: \textquestiondown Para qu'e son?, introducci'on y conceptos}
Esta secci'on se basa en los surveys:~\cite{recsys:alban,start:candillier09,recsys:nlathia}.

\medskip
Los primeros art'iculos sobre el tema aparecieron a mediados de los noventa. Estos sistmesa tienen por objetivo ayudar a los usuarios a encontrar items que sean de su agrado de entre numerosos art'iculos en un cat'alogo. Los items pueden ser de cualquier tipo: pel'iculas, m'usica, libros, sitios web, noticias, restaurantes, estilos de vida. Los sistemas de recomendaci'on ayudan a los usuarios a encontrar items de su inter'es bas'andose en alguna informaci'on sobre sus preferencias hist'oricas ~\cite{start:candillier09}.

La gran cantidad de recursos existentes hace muy dif'icil que el usuario los vea todos, eso sin contar con que se genera informaci'on nueva constantemente. Los sistemas de recomendaci'on son herramientas para la ayuda de toma de decisiones sin necesitar una b'usqueda. 

Tareas comunes para los sistemas de recomendaci'on pueden ser: encontrar buenos items, encontrar todos los items, recomendar una secuencia de items o como ayuda de navegaci'on. Sin embargo, el objetivo principal es filtrar contenido para proporcionar sugerencias relevantes y 'utiles para cada usuario del sistema. Los sistemas de recomendaci'on difieren de la tradicional recuperaci'on de informaci'on construyendo modelos para las preferencias del usuario y combinar selectivamente las opiniones de usuarios distintos para generar recomendaciones 'unicas por usuario.~\cite{recsys:nlathia}.

Existen dos tipos de problemas con los datos:
\begin{itemize}
\item Recuperaci'on de informaci'on (IR): contenido est'atico, query din'amico $\rightarrow$ modelado de contenido (organizado con 'indices).
\item Filtrado de informaci'on (IF): contenido din'amico, query est'atico $\rightarrow$ modelado de query (organizado como filtros).

La recomendaci'on se encuentra entre IR e IF ya que el contenido var'ia suavemente y los queries dependen de pocos par'ametros. Los m'etodos para IR e IF son entonces usados para reducir el proceso en tiempo de query~\cite{recsys:alban}.
\end{itemize}

\smallskip
\textbf{Usuario:} el usuario final del sistema, a quien se le quiere recomendar algo.

\smallskip
\textbf{Rating:} generar recomendaciones a menudo se describe como el problema de predecir qu'e tanto le gustar'a al usuario o la calificaci'on exacta que el usuario le dar'a a un item espec'ifico. Los ratings pueden ser expl'icitos o impl'icitos.

\smallskip
\textbf{Perfil:} Los usuarios pueden ser modelados de acuerdo a una gran variedad de informaci'on. La informaci'on m'as importante es el conjunto de evaluaciones (ratings) que han dado al sistema, los cuales corresponden a cada perfil de usuario. 

\smallskip
\textbf{Relaci'on entre ratings y perfiles:} el punto central de un sistema de recomendaci'on es el conjunto de perfiles de usuario. Los perfiles contienen una colecci'on de juicios o calificaciones del contenido disponible y proporciona una fuente invaluable de informaci'on que puede usarse para dar recomendaciones a los usuarios.

Los juicios humanos sin embargo, pueden provenir de dos fuentes separadas. Esto est'a relacionado con la amplia categor'ia de \textit{retroalimentaci'on relevante} de las t'ecnicas de recuperaci'on de informaci'on (IR). 

\begin{itemize}
\item Ratings expl'icitos. Calificaciones num'ericas dadas por el usuario.
\item Ratings impl'icitos. Son dados por el comportamiento del usuario como por ejemplo, tiempo leyendo una p'agina, n'umero de veces que escucha una canci'on o artista o el art'iculo que ha visto al navegar por un cat'alogo.
\end{itemize}

Los ratings impl'icitos pueden convertirse a un valor num'erico mediante una funci'on apropiada (transpose function). El conjunto de juicios por usuario, comparado al n'umero total de items que pueden calificarse ser'a muy peque~no.

A un conjunto de perfiles de usuario se le conoce como matriz de calificaci'on de usuario.

%-------------------------------------------------------
\medskip
\textbf{ParaPost} \\
Pedir y dar recomendaciones son cosas que seguramente se han realizado desde que el hombre apareci'o en la Tierra. Ya me imagino:

\begin{itemize}
\item Mejores 'areas de caza.
\item D'onde hay cuevas, las mejores, las que protegen m'as.
\item D'onde hay r'ios, lagos o lagunas.
\item Grupos de personas pac'ificas o conflictivas.
\end{itemize}

No ser'ia raro que las pinturas rupestres sean formas primitivas de sistemas de recomendaci'on.

La recomendaci'on de persona a persona es muy com'un y sigue y seguir'a vigente. Normalmente pedimos recomendaciones a personas con las que coincidimos en algo y sabemos que nos puede hacer recomendaciones que nos agraden.

Pero, cuando la b'usqueda es en contenidos de internet y se cuenta con mucha informaci'on es cuando los sistemas de recomendaci'on pueden ser de gran ayuda. Desde algo tan simple como \textquestiondown qu'e pel'icula ver'e hoy? hasta informaci'on sobre salud. Con el continuo y r'apido aumento de la cantidad de informaci'on, los sistemas de recomendaci'on han surgido como una  herramienta para tomar decisiones. Los recsys filtran contenido para proporcionar sugerencias 'utiles al usuario.

\medskip
\textbf{\textquestiondown C'omo se generan las recomendaciones? Tipos}. Bas'andose en c'omo se generan las recomendaciones, existen tres tipos de sistemas de recomendaci'on m'as com'unmente implementados: filtrado basado en contenido, filtrado colaborativo y filtrado h'ibrido. El que predomina es el filtrado colaborativo.

%%%%%%%%%%%%%%%%%%%%%%%%%%%%%%%%%%%%%%%%%%%%%%%%%%%%%%%%
\section{Filtrado basado en contenido}
Estos sistemas hacen un match entre un item y el usuario. Lo hacen bas'andose en la descripci'on del item y el perfil del usuario.

La idea es tener una forma de describir el item, el chiste es descomponer el contenido en atributos enumerables, variables descriptivas bien definidas. Estos atributos pueden ser frecuencia de la palabra o etiquetas que el usuario pone. Una forma de construir el perfil para que incluya los tipos de item que le gustan y una forma de comparaci'on entre item y el perfil. Los items que tengan un alto nivel de cercan'ia a las preferencias del usuario se recomendar'an. 

El \textbf{perfil de usuario} puede construirse de forma \textit{impl'icita} a partir de las preferencias del usuario para items, ya sea buscando los items que le han gustado o disgustado o en sus acciones pasadas (historial de compras). Tambi'en de forma \textit{expl'icita} mediante cuestionarios acerca de las descripciones del item.

Un \textbf{modelo del usuario} puede aprenderse de forma \textbf{impl'icita} usando un m'etodo de aprendizaje autom'atico, tomando como entrada las descripciones del item y generando como salida las apreciaciones del usuario sobre el item. 

Los perfiles del usuario generalmente se representan como vectores de pesos sobre las descripciones el item. Las recomendaciones son generadas aplicando m'etodos al modelo de preferencias del usuario. Entre estos m'etodos est'a la inducci'on de reglas y 'arboles de decisi'on.

Las preferencias indican la relaci'on entre un usuario y los datos. La cobertura de una preferencia se relaciona directamente a la cobertura del o los atributos a los cuales se aplica. Un atributo tiene una alta cobertura cuando aparece en muchos items y una baja cobertura si aparece en pocos. 

Sin embargo, la cobertura puede extenderse usando la noci'on de similaridad entre atributos. Si la preferencia de un usuario es \textit{Me gusta Jackie Chan como actor}, la cobertura es baja pero si se incluye \textit{Me gusta Jackie Chan como actor} como son considerados de forma similar, la cobertura se extiende, esto suponiendo que no existen contradicciones con otras preferencias.

 Diversos enfoques se han seguido para determinar la similaridad entre atributos. Tradicionalmente 'esto es realizado por un experto. Esto es popular para dominios peque~nos pero para grandes, es impr'actico. Como alternativa, hay medidas de similaridad para tomar ventaja de la riqueza de informaci'on existente en internet. Una de ellas es la NormalisedGoogleDistance, que infiere similaridades entre t'erminos textuales usando co-ocurrencia en websites. Sin embargo, para grandes bases de datos esto tambi'en es d'ebil.

Para evitar estas restricciones, se han preferido las m'etricas de similaridad que analizan directamente las bases de datos de los sistemas de recomendaci'on. Por ejemplo, se crean vectores para dos actores y se les aplica la medida wCosine para compararlos.

Otros enfoces son usando clasificadores como Naive Bayes, dando como entrada las descripciones del item y como salida los gustos del usuario para un subconjunto de items. El clasificador se entrena sobre un conjunto de items ya considerados por el usuario. As'i es capaz de predecir si un nuevo elemento le gustar'a o no al usuario.

Ventajas:
\begin{itemize}
\item  Usando aprendizaje inductivo con 'arboles y reglas se pueden adaptar r'apidamente y cambiar recomendaciones bas'andose en la retroalimentaci'on del usuario. Esto lleva a la idea de recomendadores conversacionales que permiten a los usuarios revisr las preferencias que dan como entrada criticando los resultados obtenidos. Los modelos del usuario son din'amicos y permiten a los usuarios entender el efecto de sus preferencias en las recomendaciones recibidas. 
\item Cubre limitaciones de los colaborativos: pueden generar recomendaciones para nuevos items sin necesidad de ratings disponibles.
\item Pueden manejar situaciones donde los usuarios no consideran los mismos items pero si items parecidos.
\end{itemize}

Desventajas:
\begin{itemize}
\item Necesitan descripciones ricas y completas de items y perfiles de usuario bien construidos. Esta es la principal limitaci'on.
\item Sufren de sobre-especializaci'on, i.e., a menudo recomiendan items de contenido similar a los items ya considerados lo que puede llevar a una falta de originalidad.
\item Las decisiones de los usuarios van mas all'a de lo que puede representarse en t'erminos de atributos.
\item No se pueden aplicar a todo el rango de escenarios.
\item Requieren que el contenido pueda ser descrito en t'erminos de atributos cosa que no siempre es posible.
\item A menudo se requieren grandes cantidades de detalles del usuario para hacer buenas recomendaciones.
\end{itemize}


%%%%%%%%%%%%%%%%%%%%%%%%%%%%%%%%%%%%%%%%%%%%%%%%%%%%%%%%
\section{Filtrado colaborativo}
Seg'un~\cite{recsys:nlathia}:\\
Los algoritmos de filtrado colaborativo, a diferencia de los sistemas basados en contenido, ignoran totalmente cualquier descripci'on o atributos de los datos. Favorecen los juicios humanos y se enfocan en generar las recomendaciones con base en las opiniones expresadas por una comunidad de usuarios. Han sido usados ampliamente en una gran diversidad de sitios web.

La generaci'on de recomendaciones y el uso de datos disponibles se ha abordado desde distintas perspectivas. Cada una aplica diferentes heur'isticas y metodolog'ias para crear recomendaciones. Se revisan dos grandes categor'ias de filtros: basado en la memoria y basado en el modelo, posteriormente se da un vistazo a otros m'etodos y enfoques h'ibridos.


%%%%%%%%%%%%%%%%%%%%%%%%%%%%%%%%%%%%%%%%%%%%%%%%%%%%%%%%
\subsection{Basado en memoria}
A menudo es referido como el m'etodo dominante para generaci'on de recomendaciones. Su clara estructura junto con sus buenos resultados lo hace una f'acil selecci'on. Se le llama basado en memoria porque asume que los usuarios que han pensado de forma parecida anteriormente, continuar'an compartiendo sus intereses en el futuro. Por lo tanto, las recomendaciones para un usuario pueden generarse prediciendo ratings de contenido no calificado, bas'andose en una agregaci'on de ratinga dados por usuarios parecidos (o cercanos) de la misma comunidad. Por esta raz'on, al proceso se le conoce como kNN, o filtro de los k vecinos m'as cercanos y consiste en tres fases: formaci'on del vecindario, agregaci'on de la opini'on y recomendaci'on.

\smallskip
\textbf{Formaci'on del vecindario:} La idea es encontrar un subconjunto 'unico de la comunidad para cada usuario, esto se hace identificando a otros usuarios con intereses similares que puedan actuar como recomendadores. Para hacerlo, cada pare de perfiles de usuario se compara para medir el grado de similaridad $w_{a,b}$ compartido. En general, el rango de similaridad var'ia desde $1$ (perfecta similaridad) a $-1$ perfecta disimilaridad. Si un par de usuarios no tiene coincidencias, no hay forma de comparar su similaridad la cual se pone a $0$.

La similaridad puede medirse de diferentes maneras. El objetivo es modelar la relaci'on potencial entre usuarios con un valor num'erico.  Veamos varias:

\begin{itemize}
\item La medida m'as simple para medir la fuerza de la relaci'on es contar la proporci'on de items co-calificados compartidos por un par de usuarios: poner ecuaci'on.

Esta medida ignora los valores de las calificaciones y s'olo considera lo que cada usuario ha calificado; es el tama~no de la intersecci'on de los dos perfiles del usuario sobre el tama~no de la uni'on. La suposici'on es que dos usuarios que cont'inuamente califican los mismos items comparten esa caracter'istica com'un.
\item El m'etodo m'as citado para medir la similaridad es el Coeficiente de Correlaci'on de Pearson que mide el grado de linearidad que existe en la intersecci'on de un par de usuarios. poner ecuaci'on.
\item Una mejora al Coeficiente de Pearson es el pesado de significancia: si el n'umero de items co-calificados $n$ es menor que un umbral $x$, la medida de similaridad es multiplicada por $n/x$. La medida es m'as confiable conforme el n'umero de items co-calificados aumenta.
\item Otra modificaci'on es el Coeficiente de Pearson con restricciones, que reemplaza la media del usuario en la ecuaci'on con el \textit{rating scale midpoint}.
\item En el pasado se han usado otras medidas de similaridad, una de ellas es la \textit{Spearman Rank Correlation}.
\item Vector Similarity (o $cosine angle$ entre dos perfiles de usuario.
\item Distancia Euclidiana y Manhattan.
\item Otros m'etodos que tratan de capturar la proporci'on de acuerdo entre usuarios tales como los de Atresty y Winner (1997).
\end{itemize}

Las medidas de similaridad permanecen como un 'area abierta ya que no se puede hacer m'as que comparar la exactitud de la predicci'on para demostrar que una medida es mejor que otra en un conjunto particular.

\smallskip
\textbf{Agregaci'on de opini'on:} una vez que las comparaciones entre el usuario y el resto de la comunidad se terminaron, tenemos un conjunto de pesos de los recomendadores por lo que se pueden predecir ratings de contenido no calificado. Al igual que la fase anterior, hay diversas formas de calcular estas predicciones. En dos m'etodos (Herlocker et al, 1999 y Koren, 2007) se predice un rating $p_{a,j}$ del item $i$ para el usuario $a$ como el promedio pesado de ratings de vecinos $r_{b,i}$. Los pesos $w_{a,b}$ son la medida de similaridad del paso anterior por lo que los vecinos m'as similares tendr'an mayor influencia en la predicci'on.

\smallskip
\textquestiondown Qu'e ratings son los escogidos para contribuir a predecir el rating? Y nuevamente hay mucho de donde escoger que tendr'a un impacto directo en los resultados. En algunos casos, se toman s'olo los $k$-mejores vecinos cercanos para contribuir. Sin embargo, a menudo ninguno de esos vecinos ha calificado al item en cuesti'on y as'i, la cobertura de la predicci'on se impacta negativamente. Una alternativa directa, por tanto, es considerar los $k$-mejores recomendadores que disponen de rating para el item en cuesti'on. Por un lado, este m'etodo garantiza que se har'a una predicci'on; por el otro las predicciones se hacen con base en usuarios modestamente similares por lo que no pueden ser muy precisas.

Una 'ultima alternativa es seleccionar s'olo los usuarios sobre un umbral de similaridad pre-determinado. Pero, \textquestiondown cu'al ser'ia el umbral o valor de $k$?. Son preguntas sin respuesta y dependen del conjunto de datos.

\smallskip
\textbf{Recomendaci'on:} una vez que se han predicho los ratings para los items y ordenados de acuerdo al valor de la predicci'on, los $n$-mejores items se proponen al usuario final como recomendaciones. Ahora, se puede obtener retroalimentaci'on del usuario. Los perfiles del usuario crecer'an y el recsys puede empezar repetir el proceso: re-calcular medidas de similaridad, predecir ratings y dar recomendaciones. Hasta el momento se ha considerado el proceso de generaci'on de recomendaciones 'unicamente por un enfoque basado-en-memoria de vecinos-cercanos. En otra secci'on se ver'an las contribuciones desde el 'area de aprendizaje autom'atico, a menudo llamadas filtrado colaborativo basado-en-el-modelo.

\medskip
De acuerdo a~\cite{start:candillier09}:\\
La entrada al sistema es un conjunto de ratings sobre los items. Los usuarios pueden compararse con base en la apreciaci'on que comparten sobre los items, creando la noci'on de vecindarios de usuarios. De forma parecida, los items pueden compararse con base en la apreciaci'on compartida por los usuarios, formando la noci'on de vecindarios de items. Los ratings del item para un para un usuario dado  pueden predecirse con base en los ratings dados en su vecindario de usuarios y el vecindario de items.

Formalizaci'on:\\
Sea $U$ un conjunto de $N$ usuarios, $I$ un conjunto de $M$ items, y $R$ un conjunto de ratings $r_{ui}$ de usuarios $u \in U$ en el item $i \in I$. $S_u \subseteq I$ significa el conjunto de items que el usuario $u$ ha calificado.

El objetivo de los enfoques de filtrado colaborativo es predecir el rating $p_{ai}$ de un usuario $a$ sobre un item $i$. Se supone que el usuario $a$ es activo, i.e., ya ha calificado algunos items. El item a ser predecido es desconocido al usuario $i \notin S_{a}$.

\smallskip
\textbf{Enfoque basado en el usuario.} Es igual que el que describe~\cite{recsys:nlathia} en la fase de \textbf{agregaci'on de opini'on} y consiste en predecir el rating de un usuario para un item con base en los vecinos cercanos.

\smallskip
\textbf{Enfoque basado en el item.} Recientemente ha crecido el inter'es por los enfoques basados en el item. Dada una medida de similaridad entre items, tales enfoques primero definen vecindades de items. La predicci'on del rating de un usuario por un item se deriva de los ratings del usuario en los vecinos del item fijado.

As'i como en los enfoques basados en el usuario, la vecindad de items tama~no $K$ es un par'ametro del sistema que necesita definirse. Dado $T_i$, la vecindad del item $i$, se consideran dos formas para predecir nuevos ratings del usuario:

\begin{enumerate}
\item usando una suma pesada
\item usando una suma pesada de las desviaciones de la media de los ratings del item
\end{enumerate}


%-------------------------------------------------------
\subsection{Basado en el modelo}
De acuerdo a~\cite{recsys:nlathia}\\
En el resumen: pag. 6, en lathia pag. 8, en Candillier pag. 4

Medidas de similaridad: correlaci'on, coseno, Manhattan, Jaccard

%-------------------------------------------------------
\section{Filtrado h'ibrido}

%%%%%%%%%%%%%%%%%%%%%%%%%%%%%%%%%%%%%%%%%%%%%%%%%%%%%%%%
\section{Interfaz de usuario}

%%%%%%%%%%%%%%%%%%%%%%%%%%%%%%%%%%%%%%%%%%%%%%%%%%%%%%%%
\section{Problemas}

%%%%%%%%%%%%%%%%%%%%%%%%%%%%%%%%%%%%%%%%%%%%%%%%%%%%%%%%
\section{Evaluaci'on de sistemas de recomendaci'on: he aqu'i el punto}

%%%%%%%%%%%%%%%%%%%%%%%%%%%%%%%%%%%%%%%%%%%%%%%%%%%%%%%%
\section{Tendencias}

%%%%%%%%%%%%%%%%%%%%%%%%%%%%%%%%%%%%%%%%%%%%%%%%%%%%%%%%
\section{Prototipo}
Prototipo~\cite{rep1:isra} y tambi'en~\cite{rep2:isra}

%%%%%%%%%%%%%%%%%%%%%%%%%%%%%%%%%%%%%%%%%%%%%%%%%%%%%%%%


%%%%%%%%%%%%%%%%%%%%%%%%%%%%%%%%%%%%%%%%%%%%%%%%%%%%%%%%
\bibliographystyle{apalike}
\bibliography{ea}
\end{document} 

%%%%%%%%%%%%%%%%%%%%%%%%%%%%%%%%%%%%%%%%%%%%%%%%%%%%%%%%
% \begin{figure}[h]
% 	\centering
% 	\includegraphics[keepaspectratio,width=2cm]{theimg/bp}
% 	\caption[Bletchley Park]{Bletchley Park} 
% 	\label{fig:bp}
% \end{figure}
% 
% 
% \begin{figure}[h]
% 	\begin{center}
% 	\subfigure[Teletipo]{\includegraphics[width=2.5cm,keepaspectratio]{theimg/teletipo}}  
% 	\hspace{1cm}
% 	\subfigure[Colossus]{\includegraphics[keepaspectratio,height=2cm]{theimg/colossus}} 
% 	\caption{Teletipo y Colossus}
%   \label{fig:teletipocolossus}
% 	\end{center}
% \end{figure} 
